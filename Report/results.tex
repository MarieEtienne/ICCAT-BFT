
\subsection{Management parameters}
\subsubsection{Fishing mortality}





\subsubsection{Spawning biomass}




\subsubsection{Kobe Plot}

{\centering \includegraphics[width=\maxwidth]{figure/ICCAT-KobePlot} 
}


%{\centering \includegraphics[width=\maxwidth]{figure/ICCAT-RetroSpawning} 

%}


%{\centering \includegraphics[width=\maxwidth]{figure/ICCAT-RetroBmsy} 

%}



%{\centering \includegraphics[width=\maxwidth]{figure/ICCAT-RetroFmsy} 
%}

\subsection{Monte Carlo approach}
\subsubsection{Prior specification}
Monte Carlo approach require to specify adequat prior on leading parameters.
\paragraph{Prior on $R_0$, $R_{init}$ and $\bar{R}$}
The prior on $\R_0$ is build with the idea of a range between $10^5$ et $10^6$ for the unfished total biomass. $\Phi_E$ being the contribution of a recruit to the total biomass during its life, 
$B_0=R_0\Phi_E$. $\Phi_E$ depends on the weight at age and the survivorship, both being fixed. Therefore $\R_0$ may be derived and the corresponding value for the logarithm of $R_0$ is in ethe interval $[13.68, 15.98]$. Relaxing this assumption, 
the prior has been specified as in equation \ref{eq:priorR0}.
\begin{equation}
\log(R_0) \sim \mathcal{U}[12, 17] 
\label{eq:priorR0}
\end{equation}
The same priors have been chosen for $R_{init}$ and $\bar{R}$. 


\paragraph{Prior for steepness $h$}
A beta distribution has been chosen as a prior for the steepness. The parameters of the beta distribution has been chosen so that the mode of the steepness is $0.9$ and the coefficient of variation is 
$10\%$ which to the distribution specified in equation \ref{eq:priorh}.
\begin{equation}
(R_0) \sim \beta\left(14, 2.44\right)
\label{eq:priorh}
\end{equation}



\subsubsection{Results}
\paragraph{Convergence}
{\centering \includegraphics[width=\maxwidth]{figure/ICCAT-Conv1} 
{\centering \includegraphics[width=\maxwidth]{figure/ICCAT-Conv2}

\paragraph{Stock Statut}
{\centering \includegraphics[width=\maxwidth]{figure/ICCAT-PostStatus} 
}

\subsection{Comparaison with results from stock assesmnent}

\subsection{Sensitivity analysis}