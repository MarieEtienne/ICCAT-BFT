
\subsection{Tunning up \iscam - choices}
%\ASS{The first age class is 1 year, recruitment at year $t$ depends on
%  mature at year $t-1$, according to ICCAT report}
%\ASS{Beverton-Holt model is used for recruitment -- may be changed }

\ASS{To account  for the commercial  catch (only one time  series and
for the seven abundance indices),  eight gears have been declared. gear
one corresponds  to commercial fisheries,  gear 2 to 8  corresponds to
the seven abundance indices, therefore $F_{f,t}=0$ for $k \geq 2$}

\paragraph{Mortality}
The mortality at age is fixed to the values used for the VPA approach.

\begin{table}[ht]
\centering
\begin{tabular}{rrrrrrrrrrr}
  \hline
 & 1 & 2 & 3 & 4 & 5 & 6 & 7 & 8 & 9 & 10 \\ 
  \hline
Mortality & 0.49 & 0.24 & 0.24 & 0.24 & 0.24 & 0.20 & 0.17 & 0.15 & 0.12 & 0.10 \\ 
   \hline
\end{tabular}
\end{table}


\paragraph{Selectivity}
As mentionned in \iscam allows to specify different form for the
selectivity/vulnerability.  When  age composition data  are available,
choice has been made to model  the selectivity using Bspline curves or
logistic function.



% ## ## SELECTIVITY PARAMETERS Columns for gear                                   ##
% ## ## OPTIONS FOR SELECTIVITY (isel_type):                                      ##
% ## 3 1 1 6 1 3 13  # 1  -selectivity type ivector(isel_type) for gear



The numbers  of nodes for the  Bspline curves can be  chosen equal for
all gear with age composition data 

% \begin{knitrout}
% \definecolor{shadecolor}{rgb}{0.969, 0.969, 0.969}\color{fgcolor}\begin{kframe}
% \begin{verbatim}
% ## 5 0 0 0 0 5 0  # 4  -No. of age nodes for each gear (0=ignore)
% \end{verbatim}
% \end{kframe}
% \end{knitrout}

For gear  4 corresponding to the  Norvegian Purse seine index,  no age
composition data  are available.  In \cite{tuna2012}, it  is indicated
that this index is  relevant only for the last class  age. Since it is
not  directly   possible  to  specify   a  selectivity  with   only  0
coefficient except for one age in \iscam, a logistic selectivity curve has been
adopted with the following parameters.
% \begin{knitrout}
% \definecolor{shadecolor}{rgb}{0.969, 0.969, 0.969}\color{fgcolor}\begin{kframe}
% \begin{verbatim}
% ## 6 6 6 9.9 6 6 6  # 2  -Age/length at 50% selectivity (logistic)
% ## 1 1 1 0.1 1 1 1  # 3  -STD at 50% selectivity (logistic)
% \end{verbatim}
% \end{kframe}
% \end{knitrout}


\paragraph{Abundance indices}



The total vulnerable biomass at year $t$ for gear $k$ is defined by
\begin{gather}
V_{k,t}=\sum_{a=1}^A N_{t,a} e^{-\lambda_{k,t} Z_{t,a}} v_{k,a} w_a,
\end{gather}
 where $\lambda_{k,t}$ is the fraction  of the mortality to adjust for
 survey timing, it is specified by the user. They have been
 specified according to the input data file 
% \begin{knitrout}
% \definecolor{shadecolor}{rgb}{0.969, 0.969, 0.969}\color{fgcolor}\begin{kframe}
% \begin{verbatim}
% ## ## Survey timing (if 0, the gear has no associated index)
% ## 0	 0.5	0.5	0.5	0.0833333333333333	0.5	0.5
% \end{verbatim}
% \end{kframe}
% \end{knitrout}


\paragraph{Recruitment}


\paragraph{Life trait specification}

$$\Phi = (l_\infty, k, t_o,a,b,\dot{a},\dot{\gamma})$$
\begin{verbatim}
## linf  =  319
## k  =  0.093
## to  =  -0.97
##  sclw = 1.95e-05 #1.95e-5 #scaler in length-weight allometry
## plw =  3.009  #power in length-weight allometry
## m50 =  4 #50% maturity
## std50 =  0.8 #std at 50% maturity
\end{verbatim}




\subsubsection{Initial values for estimated parameters}
$$\theta   =  (R_0,   \kappa,   M,  \bar{R},   \rho,  v^2,   \gamma_k,
\boldsymbol{F}_{t}, (\phi_t)_{t=1}^T, (\epsilon_t^R)_{t=1}^T)$$


	%% ro          = mfexp(theta(1));
	%% dvariable h = theta(2);
	%% m           = mfexp(theta(3));
	%% log_avgrec  = theta(4);
	%% log_recinit = theta(5);
	%% rho         = theta(6);
	%% varphi      = sqrt(1.0/theta(7));
	%% sig         = sqrt(rho) * varphi;
	%% tau         = sqrt(1-rho) * varphi;

        
        The  control  file  provides  initial  values  for  parameters
        $\theta$, the  following values  have been used  to initialise
        the model.

\paragraph{Value for $R_0$}


\paragraph{Value for $\bar{R}$}
If the  model doesn't supposed  unfished conditions at  starting year,
the recruitment at  the first year is  not $R_0$. It is  assumed to be
equal to
$$R_{t=1} = \bar{R}_{init} \exp{\epsilon_1^R}$$
In \iscam code, the recruitmnent at year $t$ is defined by
$$R_t(t)=\bar{R} \exp{\epsilon_t^R}, $$
  where $\epsilon_t^R\overset{i.i.d}{\sim} \mathcal{N}(0,\sigma_R^2)$.


\subsection{Assumption on vulnerability parameters}
When catch at age data are available, a vunlnerability curve using cubic Bsplines is fitted and currently it is assumed to be constant over time. (\com{Option 3 in selectivity option for \iscam})
It is currently not possible with \iscam to specify 0 for vulnerability, since it is expressed in log scale and because \admb behaves better with differentiable functions.





The file used for illustrating the way to use iscam on Bluefin Tuna is
\begin{verbatim}
## /home/metienne/ICCAT/ICCAT-BFT/Inputs/bfte/2012/vpa/reported/low/bfte2012.d1
\end{verbatim}

{\centering \includegraphics[width=\maxwidth]{figure/ICCAT-Selectivity2} 

}

Only 6 gears


{\centering \includegraphics[width=\maxwidth]{figure/ICCAT-SelectivityBygear} 
}




\begin{figure}
{\centering \includegraphics[width=\maxwidth]{figure/ICCAT-SelecBef80} 

}

\caption[Selectivity at age before 1980]{Selectivity at age before 1980\label{fig:SelecBef80}}
\end{figure}


\begin{figure}
{\centering \includegraphics[width=\maxwidth]{figure/ICCAT-SelecAft80} 

}
\caption[Selectivity at age after 1980]{Selectivity at age after 1980\label{fig:SelecAft80}}
\end{figure}


\subsection{First results}
\begin{verbatim}
## [1] 0.1385
\end{verbatim}
\definecolor{shadecolor}{rgb}{0.969, 0.969, 0.969}\color{fgcolor}
\begin{alltt}
\hlcomment{# print(res$A) print(res$Ahat) print(res$A_nu)}
\end{alltt}
\begin{alltt}
\hlfunctioncall{print}(res$fmsy)
\end{alltt}
\begin{alltt}
\hlfunctioncall{print}(res$msy)
\end{alltt}
\begin{verbatim}
## [1] 29631000
\end{verbatim}
\begin{alltt}
\hlfunctioncall{print}(res$bmsy)
\end{alltt}
\begin{verbatim}
## [1] 152605000
\end{verbatim}
\begin{alltt}
\hlfunctioncall{print}(res$bo)
\end{alltt}
\begin{verbatim}
## [1] 526851000
\end{verbatim}
\begin{alltt}
\hlfunctioncall{print}(res$ro)
\end{alltt}
\begin{verbatim}
## [1] 2617200
\end{verbatim}
\begin{alltt}
\hlfunctioncall{print}(res$q)
\end{alltt}
\begin{verbatim}
## [1] 7.508e-04 7.771e-07 2.755e-08 1.903e-07 1.642e-06 6.104e-06
\end{verbatim}


\begin{alltt}
\hlfunctioncall{print}(res$steepness)
\end{alltt}
\begin{verbatim}
## [1] 0.9083
\end{verbatim}

\paragraph{Selectivity}

The vulnerability matrix is estimated to 
\begin{resultz}
V=(v_{k,a})=\left(
    \begin{matrix}
      0.64 & 0.87 & 1.34 & 1.31 & 0.91 & 0.73 & 0.79 & 0.96 & 1.16 & 1.27 \\ 
      0.00 & 0.00 & 0.01 & 0.02 & 0.06 & 0.15 & 0.41 & 1.07 & 2.64 & 5.64 \\ 
      0.00 & 0.00 & 0.01 & 0.02 & 0.06 & 0.18 & 0.52 & 1.36 & 2.97 & 4.85 \\ 
      0.00 & 0.00 & 0.00 & 0.00 & 0.00 & 0.00 & 0.00 & 0.00 & 0.00 & 9.97 \\ 
      0.10 & 0.34 & 0.79 & 1.13 & 1.25 & 1.27 & 1.28 & 1.28 & 1.28 & 1.28 \\ 
      2.05 & 2.20 & 2.29 & 1.75 & 0.98 & 0.45 & 0.18 & 0.06 & 0.02 & 0.01 \\ 
      2.61 & 2.44 & 2.05 & 1.45 & 0.81 & 0.38 & 0.15 & 0.06 & 0.02 & 0.01 \\ 
    \end{matrix} 
  \right)
  \label{res:seltable}
\end{resultz}
