The first section  aims at clarifying the choices made  to run \iscam
on BFT  data. The second section  shows the estimation of  the leading
parameters in  order to  justify the  simulation framework  develop in
section \ref{sec:simulation}. The detailed  complete results are given
in section \ref{sec:results}.




\subsection{Tunning up \iscam - choices}

\paragraph{Preliminary remarks}
The first age class is 1 year, recruitment at year $t$ depends on
mature  biomass  at  year  $t-1$,  according  to  \cite{tuna2012}.  The
Beverton-Holt model is used for recruitment.


To account for the total commercial catch (only one time series) and
for the six abundance indices considered,  seven gears have been declared. Gear
one corresponds  to commercial fisheries,  gears 2 to 7  correspond to
the six  abundance indices, each  gear having its  one vulnerability
specification. The fishing mortality is  driven by the catch for gear
1 and  therefore $F_{k,t}=0$ for  $k \geq  2$. For clarity,  index $k$
will be omitted in the sequel since there is no ambiguity.

\paragraph{Mortality rate}
In \cite{tuna2012}, the mortality rate is defined by 
\begin{table}[ht]
\centering
\begin{tabular}{rrrrrrrrrrr}
  \hline
 & 1 & 2 & 3 & 4 & 5 & 6 & 7 & 8 & 9 & 10 \\ 
  \hline
Mortality & 0.49 & 0.24 & 0.24 & 0.24 & 0.24 & 0.20 & 0.17 & 0.15 & 0.12 & 0.10 \\ 
   \hline
\end{tabular}
\end{table}


But \iscam doesn't allow to use  instantaneous mortality at age whereas  it is possible to use a time varying mortality, using a random walk centered on a given value.
Because of the well known difficulty for estimating the natural mortality rate, it has been decided to use a fixed instantaneous mortality rate.
This parameter won't be estimated and is fixed to 0.23.


\paragraph{Selectivity}
As mentioned in \cite{Martell12},	 \iscam allows to specify different form for the
selectivity/vulnerability.  When age composition data are available,
choice has been made to model  the selectivity using Bspline curves or
logistic function.  The selectivity has  been chosen to  stay constant
over years. The specific choice between B-splines and logistic curves has been made to 
assure the convergence of the optimization algorithm and the obtention of a definite positive Hessian matrix.



\begin{table}[ht]
\centering
\begin{tabular}{rrp{8cm}}
  \hline
  Gear & Gear Name & Selectivity shape \\ 
  \hline
1 & \verb+Catch+ & Constant (over years) cubic B-spline with 5 nodes \\
2 & \verb+SM_TP+ & Logistic shape \\ 	
3 & \verb+LL_JP1+ & Logistic shape \\
4 & \verb+NW_PS+ & fixed logistic $\mu_{50}=9.9$, $sd=0.1$\\
5 & \verb+JP_LL2+ & Logistic shape \\
6 & \verb+SP_BB1+ & Constant (over years) cubic B-spline with 5 nodes \\
7 & \verb+SP_BB2+ & Constant (over years) cubic B-spline with 5 nodes \\
8 & \verb+SP_BB3+ & Logistic shape \\
\hline
\end{tabular}
\caption{Choice for the selectivity curve for each gear}
\label{table:selectivity}
\end{table}


Gear  $4$ corresponds  to the  Norwegian Purse  seine abundance  index
which is documented  to focus  only on  last age  class. Since  it is  not possible  to
specify $0$ vulnerability for some  age (due to a log transformation),
the selectivity curve  has been set to a logistic  function with fixed
parameters, the parameters being chosen to mimic the desired behavior.


\paragraph{Abundance indices}



The total vulnerable biomass at year $t$ for gear $k$ is defined by
\begin{gather}
V_{k,t}=\sum_{a=1}^A N_{t,a} e^{-\lambda_{k,t} Z_{t,a}} v_{k,a} w_a,
\end{gather}
 where $\lambda_{k,t}$ is the fraction  of the mortality to adjust for
 survey timing, it is specified by the user. They have been
 specified according to the input data file 
 
Respectively, the total vulnerable number of fish at year $t$ for gear $k$ is defined by
\begin{gather}
V^{N}_{k,t}=\sum_{a=1}^A N_{t,a} e^{-\lambda_{k,t} Z_{t,a}} v_{k,a},
\end{gather}
 
 
 The value of $\lambda_{k,t}$ are taken constant over years. Five of them are specified as average indices along the year in the stock assessment,
the last one is assumed to occur one month after the start of the year. To mimic those choices, the timing parameters $\lambda$ are equal to 0.5 for five of the six indices, the last one is set to $1/12$. Those choices are sum up in table \ref{table:timing} 
 
 \begin{table}[ht]
\centering
\begin{tabular}{rrp{3.5cm}p{4cm}}
  \hline
  Gear & Gear Name & Number/Biomass & $\lambda_k$: Timing for the survey\\ 
  \hline
2 & \verb+SM_TP+ & Number & 0.5 \\ 	
3 & \verb+LL_JP1+ & Number & 0.5 \\
4 & \verb+NW_PS+ & Biomass & 0.5 \\
5 & \verb+JP_LL2+ & Number & 1/12 \\
6 & \verb+SP_BB1+ & Biomass & 0.5\\
7 & \verb+SP_BB2+ & Biomass &0.5 \\
\hline
\end{tabular}
\caption{Choice for the type and the timing of abundance indices}
\label{table:timing}
\end{table}
 


\paragraph{Recruitment}
The estimation of the recruitment parameters in \iscam  is split into two steps. In early phases, the recruitment is assumed to be stock independent using formula 
\ref{eq:iscamRec}. In the last phases, a stock recruitment relationship is used as specified 
in equation \ref{eq:iscamBev}. 

\paragraph{Life trait history}
The parameters regarding the life trait history are not estimated and are treated as fixed when running \iscam. Actually, there is some possibility to use weight at age data to estimate  the weight at age relationship, but trying to use this option leads to inconsistent results.
The values of the concerned parameters have been fixed according to \cite{tuna2012} and are summarized in table \ref{table:lifetraits}.

\begin{table}[ht]
\centering
\begin{tabular}{p{2.5cm}p{3cm}p{3cm}p{6cm}}
  \hline
Parameters & Name in \iscam & Set Value & Signification  \\ \hline
 $l_\infty$ & &$319$ & vonB parameters\\
 $k$ & & $0.093$&vonB parameters\\
 $t_0$ & &$-0.97$ &vonB parameters\\
 $a_w$ & a &$1.95e-05$  & Weight at age allometric parameter\\
 $b_w$ & b &  $3.009$ &Weight at age allometric parameter\\
 $\mu_f$& $\dot{a}$ & $4$ & age for 50\% maturity\\
 $\sigma_f$ & $\dot{\gamma}$ &  0.8 & Standard deviation at 50\% maturity\\
$\Phi_E$ &  &  $201$& incidence function\\
\hline
\end{tabular}
\caption{The parameters reported in this table are considered as fixed to the specified value}
\label{table:lifetraits}
\end{table}






\subsubsection{Initial values for estimated parameters}

The leading parameters are given in table \ref{tab:parameters}. \iscam requires to provide initial values to the parameters to start the optimization algorithm.
Since the role of the recruitment differs between first phases and the last ones, it is also required to give initial values for $R_{init}$ and $\overline{R}$.
The proportion of the total variance allocated to the error in the observation process is not estimated and is set to $\rho=0.4$.

\begin{table}[ht]
\centering
\begin{tabular}{p{2.1cm}p{2.1cm}p{2.2cm}p{2cm}p{6cm}}
  \hline
Parameters & Initial Value & interval & Phase & Signification  \\ \hline
 $\log{R_0}$ & 13  & $[-5,30]$ & 1 & Logarithm of recruitment (numbers) in unfished conditions\\
 $h$ & $0.85$ & $[0.2, 0.99]$ & 3 &  Steepness \\
$\log{\bar{m}}$ & $-1.47$ & $[-5,0]$ &-1 & Logarithm of the average natural mortality rate\\
$\bar{R}$& 12.5	& $[-5,	20]$ &	1	 & Logarithm of the average recruitment\\
$R_{init}$ & 12.5&	$[-5,	20]$ &	1	 & Logarithm of the initial recruitment\\
$\rho$ & 0.4	& $[0.001,0.999]$ & -1 & proportion of the total variance for the observation process \\
$1/\varphi$ & 0.8 &	$[0.001,	12]$ & 	3 & the root of the precision of the total error \\
\hline
\end{tabular}
\caption{Initial values and range for main parameters. The estimation phase is also given.}
\label{table:initial}
\end{table}


\subsection{Bayesian approach - Prior specification}
A Bayesian approach requires to specify adequate prior distributions on leading parameters.
\paragraph{Prior on $R_0$, $R_{init}$ and $\bar{R}$}
The prior on $R_0$ is build with the idea of a range between $10^5$ et $10^6$ Tons for the unfished total biomass. $\Phi_E$ being the contribution of a recruit to the total biomass during its life, 
$B_0=R_0\Phi_E$. $\Phi_E$ depends on the weight at age and the survivorship, both being fixed. Therefore $R_0$ may be derived and the corresponding value for the logarithm of $R_0$ is in the interval $[12.9, 15.2$. Relaxing this assumption, 
the prior has been specified as in equation \ref{eq:priorR0}.
\begin{equation}
\log(R_0) \sim \mathcal{U}[12, 16] 
\label{eq:priorR0}
\end{equation}
The same priors have been chosen for $R_{init}$ and $\bar{R}$. 


\paragraph{Prior for steepness $h$}
A beta distribution has been chosen as a prior for the steepness. The parameters of the beta distribution has been chosen so that the mode of the steepness is $0.9$ and the coefficient of variation is 
$10\%$. This consideration leads to the specification in \ref{eq:priorh}.
\begin{equation}
h \sim \beta\left(14, 2.44\right)
\label{eq:priorh}
\end{equation}


\paragraph{Prior for the total variance}
$\varphi^2$ being the total variance, \iscam allows to specify a prior distribution on $1/\varphi^2$ using classically a gamma distribution. The parameters of this gamma distribution has been chosen so that to define a vague prior.
\begin{equation}
1/\varphi \sim \Gamma\left(0.1, 0.1\right)
\label{eq:priorphi}
\end{equation}

\subsection{First run}
To illustrate the results obtained on BFT East Stock, we focus on one datafile: 

\verb+Inputs/bfte/2012/vpa/inflated/high/bfte2012.d1+

\begin{center}
 \begin{figure}[bt]
 \includegraphics[width=0.9\maxwidth]{figure/ICCAT-Selectivity}
 \caption{Estimated selectivity. The additional points represents the average empirical proportion at ages}
\label{fig:sel1}
\end{figure}
 \begin{figure}[bt]
 \includegraphics[width=0.9\maxwidth]{figure/ICCAT-SelectivityByGear} 
\caption{Estimated selectivity and the empirical proportion at ages.}
\label{fig:sel2}
\end{figure}

\end{center}

Figure \ref{fig:sel1} seems to present a good coherence between the empirical proportion at ages and the estimated vulnerability. Unfortunately figure \ref{fig:sel2} highlights the large variability around the average behavior.

\begin{table}[ht]
\centering
\begin{tabular}{p{2.1cm}p{4cm}p{4cm}}
  \hline
Parameters & Initial value & Reported value   \\ \hline
 $\log{R_0}$ & $13$  &  $14.75$  \\
 $h$ & $0.85$ & $0.934$  \\
$log(\bar{R})$& $12.5$ & $14.38$ \\
$R_{init}$ & $12.5$ & $15.22$\\
$\rho$ & $0.4$ &  $0.4$  	\\
$1/\varphi$ & $1.25$ & $1.09$ \\
$log(B_msy)$ &  &$18.81$  \\
$(f_msy)$ &  &$0.14$  \\
$log(B_{2011})$ &  &$19.72$ \\
$(f_{2011})$ &   &$0.037$  \\
\hline
\end{tabular}
\caption{Initial values and range for main parameters.}
\label{table:estimation}
\end{table}



This first run leads to the Kobe plot presented in figure \ref{fig:KobePlotFirstRun}.
 \begin{figure}
{\centering \includegraphics[width=\maxwidth]{figure/ICCAT-KobePlot} }
 \caption{Evolution of the fishing effort and the stock status.}
\label{fig:KobePlotFirstRun}
 \end{figure}


\paragraph{Selectivity}

The vulnerability matrix is estimated to 
\begin{resultz}
V=(v_{k,a})=\left(
    \begin{matrix}
0.063 & 0.085 & 0.134 & 0.132 & 0.092 & 0.074 & 0.079 & 0.097 & 0.117 & 0.127 \\ 
0.000 & 0.000 & 0.001 & 0.002 & 0.006 & 0.015 & 0.041 & 0.108 & 0.264 & 0.563 \\ 
0.000 & 0.000 & 0.001 & 0.002 & 0.006 & 0.018 & 0.053 & 0.137 & 0.297 & 0.486 \\ 
0.000 & 0.000 & 0.000 & 0.000 & 0.000 & 0.000 & 0.000 & 0.000 & 0.000 & 1.000 \\ 
0.010 & 0.038 & 0.086 & 0.115 & 0.123 & 0.125 & 0.125 & 0.126 & 0.126 & 0.126 \\ 
0.205 & 0.221 & 0.229 & 0.175 & 0.098 & 0.045 & 0.018 & 0.006 & 0.002 & 0.001 \\ 
0.224 & 0.204 & 0.162 & 0.122 & 0.090 & 0.066 & 0.048 & 0.035 & 0.026 & 0.023 \\ 
  \end{matrix} 
  \right)
  \label{res:seltable}
\end{resultz}
