The first section  aims at expliciting the choices made  to run \iscam
on BFT  data. The second section  shows the estimation of  the leading
parameters in  order to  justify the  simulation framework  develop in
section \ref{sec:simulation}. The detailed  complete results are given
in section \ref{sec:results}.




\subsection{Tunning up \iscam - choices}

\paragraph{Preliminary remarks}
The first age class is 1 year, recruitment at year $t$ depends on
mature  biomass  at  year  $t-1$,  according  to  \cite{tuna2012}.  The
Beverton-Holt model is used for recruitment.


To account for the total commercial catch (only one time series) and
for the seven abundance indices,  eight gears have been declared. Gear
one corresponds  to commercial fisheries,  gear 2 to 8  corresponds to
the seven  abundance indices, each  gear having its  one vulnerability
specification. The fishing moratility is  driven by the catch for gear
1 and  therefore $F_{k,t}=0$ for  $k \geq  2$. For clarity,  index $k$
will be omitted in the sequel since there is no ambiguity.

\paragraph{Mortality}
The mortality at age is fixed to  the values used for the VPA approach
described in \cite{tuna2012}.

\begin{table}[ht]
\centering
\begin{tabular}{rrrrrrrrrrr}
  \hline
 & 1 & 2 & 3 & 4 & 5 & 6 & 7 & 8 & 9 & 10 \\ 
  \hline
Mortality & 0.49 & 0.24 & 0.24 & 0.24 & 0.24 & 0.20 & 0.17 & 0.15 & 0.12 & 0.10 \\ 
   \hline
\end{tabular}
\end{table}


\paragraph{Selectivity}
As mentionned in \cite{Martell12},	 \iscam allows to specify different form for the
selectivity/vulnerability.  When age composition data are available,
choice has been made to model  the selectivity using Bspline curves or
logistic function.  The selectivity has  been chosen to  stay constant
over years. The specific choice between B-splines and logistic curves has been made to 
assure the convergence of the optimization algorithm and the obtention of a definite positive Hessian matrix.



\begin{table}[ht]
\centering
\begin{tabular}{rrp{8cm}}
  \hline
  Gear & Gear Name & Selectivity shape \\ 
  \hline
1 & \verb+Catch+ & Constant (over years) cubic B-spline with 5 nodes \\
2 & \verb+SM_TP+ & Logistic shape \\ 	
3 & \verb+LL_JP1+ & Logistic shape \\
4 & \verb+NW_PS+ & fixed logistic $\mu_{50}=9.9$, $sd=0.1$\\
5 & \verb+JP_LL2+ & Logistic shape \\
6 & \verb+SP_BB1+ & Constant (over years) cubic B-spline with 5 nodes \\
7 & \verb+SP_BB2+ & Constant (over years) cubic B-spline with 5 nodes \\
8 & \verb+SP_BB3+ & Logistic shape \\
\hline
\end{tabular}
\caption{Choice for the selectivity curve for each gear}
\label{table:selectivity}
\end{table}


Gear  $4$ corresponds  to the  Norvegian Purse  seine abundance  index
which is documented  to focus  only on  last age  class. Since  it is  not possible  to
specify $0$ vulnerability for some  age (due to a log transformation),
the selectivity curve  has been set to a logistic  function with fixed
parameters, the parameters being chosen to micmic the desired behavior.


\paragraph{Abundance indices}



The total vulnerable biomass at year $t$ for gear $k$ is defined by
\begin{gather}
V_{k,t}=\sum_{a=1}^A N_{t,a} e^{-\lambda_{k,t} Z_{t,a}} v_{k,a} w_a,
\end{gather}
 where $\lambda_{k,t}$ is the fraction  of the mortality to adjust for
 survey timing, it is specified by the user. They have been
 specified according to the input data file 
 
Respectively, the total vulnerable number of fish at year $t$ for gear $k$ is defined by
\begin{gather}
V^{N}_{k,t}=\sum_{a=1}^A N_{t,a} e^{-\lambda_{k,t} Z_{t,a}} v_{k,a},
\end{gather}
 
 
 The value of $\lambda_{k,t}$ are taken constant over years. Six of them are specified as average indice along the year in the stock assesment,
the last is assumed to happen one month after the start of the year. To micmic those choices, the timing parameters $\lambda$ are equal to 0.5 for six of the seven indices, the last one is set to $1/12$. Those choices are sum up in table \ref{table:timing} 
 
 \begin{table}[ht]
\centering
\begin{tabular}{rrp{3.5cm}p{4cm}}
  \hline
  Gear & Gear Name & Number/Biomass & $\lambda_k$: Timing for the survey\\ 
  \hline
2 & \verb+SM_TP+ & Number & 0.5 \\ 	
3 & \verb+LL_JP1+ & Number & 0.5 \\
4 & \verb+NW_PS+ & Biomass & 0.5 \\
5 & \verb+JP_LL2+ & Number & 1/12 \\
6 & \verb+SP_BB1+ & Biomass & 0.5\\
7 & \verb+SP_BB2+ & Biomass &0.5 \\
8 & \verb+SP_BB3+ & Biomass &0.5\\
\hline
\end{tabular}
\caption{Choice for the type and the timing of abundance indices}
\label{table:timing}
\end{table}
 


\paragraph{Recruitment}
The estimation of the recruitment parameters in \iscam  is split into two steps. In early phases, the recruitment is assumed to be stock independant using formula 
to follow equation \ref{eq:iscamRec} and to be stock independant. In the last phases, a stock recruitment relationship is used as specified 
in equation \ref{eq:iscamBev}. 

\paragraph{Life trait history}
The parameters regarding the life trait history are not estimated and are treated as fixed when running \iscam. Actually, there is some possibility to use weight at age data to estimate  the weight at age relatioship, but trying to use this option leads to unconsistent results.
The values of the concerned parameters have been fixed according to \cite{tuna2012} and are summarized in table \ref{table:lifetraits}.

\begin{table}[ht]
\centering
\begin{tabular}{p{2.5cm}p{3cm}p{3cm}p{6cm}}
  \hline
Parameters & Name in \iscam & Set Value & Signification  \\ \hline
 $l_\infty$ & &$319$ & vonB parameters\\
 $k$ & & $0.093$&vonB parameters\\
 $t_0$ & &$-0.97$ &vonB parameters\\
 $a_w$ & a &$1.95e-05$  & Weight at age allometric parameter\\
 $b_w$ & b &  $3.009$ &Weight at age allometric parameter\\
 $\mu_f$& $\dot{a}$ & $4$ & age for 50\% maturity\\
 $\sigma_f$ & $\dot{\gamma}$ &  0.8 & Standard deviation at 50\% maturity\\
\hline
\end{tabular}
\caption{The parameters reported in this table are considered as fixed to the specified value}
\label{table:lifetraits}
\end{table}






\subsubsection{Initial values for estimated parameters}

The leading parameters are given in table \ref{tab:parameters}. \iscam recquires to give initial values for some parameters to start the optimization algorithm.
Since the role of the recruitment differs between first phases and the last ones, it is also required to give initial values for $\R_{init}$ and $\overline{R}$.
The proportion of the total variance allocated to the error in the observation process is not estimated and is set to $\rho=0.4$.

\begin{table}[ht]
\centering
\begin{tabular}{p{2.1cm}p{2.1cm}p{2.2cm}p{2cm}p{6cm}}
  \hline
Parameters & Initial Value & interval & Phase & Signification  \\ \hline
 $\log{R_0}$ & 13  & $[-5,30]$ & 1 & Logarithm of recruitment (numbers) in unfished conditions\\
 $h$ & $0.85$ & $[0.2, 0.99]$ & 3 &  Steepness \\
$\log{\bar{m}}$ & $-1.47$ & $[-5,0]$ &-1 & Logarithm of the average natural mortality\\
$\bar{R}$& 12.5	& $[-5,	20]$ &	1	 & Logarithm of the average recruitment\\
$R_{init}$ & 12.5&	$[-5,	20]$ &	1	 & Logarithm of the initial recruitment\\
$\rho$ & 0.4	& $[0.001,0.999]$ & -1 & proportion of the total variance for the observation process \\
$1/\varphi$ & 0.8 &	$[0.001,	12]$ & 	3 & the root of the precision of the total error \\
\hline
\end{tabular}
\caption{Initial values and range for main paramaters. The estimation phase is also given.}
\label{table:initial}
\end{table}



\subsubsection{Restriction on the datafile}
Due to high numerical instability when using all abundance indices, we focus only on the six first abundance indices and ignore the indice \verb+SP_BB3+ .


\subsection{First run}
%To illustrate the results obtained on BFT East Stock, we focus on one datafile: \verb+Inputs/bfte/2012/vpa/reported/low/bfte2012.d1+


{\centering \includegraphics[width=\maxwidth]{figure/ICCAT-Selectivity2} 

}



{\centering \includegraphics[width=\maxwidth]{figure/ICCAT-SelectivityBygear} 
}


\begin{table}[ht]
\centering
\begin{tabular}{p{2.1cm}p{4cm}p{4cm}p{4cm}}
  \hline
Parameters & Inital value & Reported value & ICCAT 2012  report  \\ \hline
 $\log{R_0}$ & $13$  &  $14.63$ & \\
 $h$ & $0.85$ & $0.95$  &\\
$log(\bar{R})$& $12.5$ & $14.29$ \\
$R_{init}$ & $12.5$ & $14.42$\\
$\rho$ & $0.4$ &  $0.4$ & 	\\
$1/\varphi$ & $1.25$ & $1.017$ &\\
$log(B_msy)$ &  &$18.71$ & \\
$(f_msy)$ &  &$0.153$ & \\
$log(B_{2011})$ &  &$19.44$ & \\
$(f_{2011})$ &   &$0.047$ & \\
\hline
\end{tabular}
\caption{Initial values and range for main paramaters.}
\label{table:estimation}
\end{table}



This first run leads to the Kobe plot presented in figure \ref{fig:KobePlotFirstRun}.
 \begin{figure}
{\centering \includegraphics[width=\maxwidth]{figure/KobePlot} }
 \caption{Evolution of the fishing effort and the stock status.}
\label{fig:KobePlotFirstRun}
 \end{figure}


\paragraph{Selectivity}

The vulnerability matrix is estimated to 
\begin{resultz}
V=(v_{k,a})=\left(
    \begin{matrix}
0.063 & 0.085 & 0.134 & 0.132 & 0.092 & 0.074 & 0.079 & 0.097 & 0.117 & 0.127 \\ 
0.000 & 0.000 & 0.001 & 0.002 & 0.006 & 0.015 & 0.041 & 0.108 & 0.264 & 0.563 \\ 
0.000 & 0.000 & 0.001 & 0.002 & 0.006 & 0.018 & 0.053 & 0.137 & 0.297 & 0.486 \\ 
0.000 & 0.000 & 0.000 & 0.000 & 0.000 & 0.000 & 0.000 & 0.000 & 0.000 & 1.000 \\ 
0.010 & 0.038 & 0.086 & 0.115 & 0.123 & 0.125 & 0.125 & 0.126 & 0.126 & 0.126 \\ 
0.205 & 0.221 & 0.229 & 0.175 & 0.098 & 0.045 & 0.018 & 0.006 & 0.002 & 0.001 \\ 
0.224 & 0.204 & 0.162 & 0.122 & 0.090 & 0.066 & 0.048 & 0.035 & 0.026 & 0.023 \\ 
  \end{matrix} 
  \right)
  \label{res:seltable}
\end{resultz}
