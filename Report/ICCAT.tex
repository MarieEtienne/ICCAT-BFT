\documentclass[a4paper]{article}\usepackage{graphicx, color}
%% maxwidth is the original width if it is less than linewidth
%% otherwise use linewidth (to make sure the graphics do not exceed the margin)
\makeatletter
\def\maxwidth{ %
  \ifdim\Gin@nat@width>\linewidth
    \linewidth
  \else
    \Gin@nat@width
  \fi
}
\makeatother

\definecolor{fgcolor}{rgb}{0.2, 0.2, 0.2}
\newcommand{\hlnumber}[1]{\textcolor[rgb]{0,0,0}{#1}}%
\newcommand{\hlfunctioncall}[1]{\textcolor[rgb]{0.501960784313725,0,0.329411764705882}{\textbf{#1}}}%
\newcommand{\hlstring}[1]{\textcolor[rgb]{0.6,0.6,1}{#1}}%
\newcommand{\hlkeyword}[1]{\textcolor[rgb]{0,0,0}{\textbf{#1}}}%
\newcommand{\hlargument}[1]{\textcolor[rgb]{0.690196078431373,0.250980392156863,0.0196078431372549}{#1}}%
\newcommand{\hlcomment}[1]{\textcolor[rgb]{0.180392156862745,0.6,0.341176470588235}{#1}}%
\newcommand{\hlroxygencomment}[1]{\textcolor[rgb]{0.43921568627451,0.47843137254902,0.701960784313725}{#1}}%
\newcommand{\hlformalargs}[1]{\textcolor[rgb]{0.690196078431373,0.250980392156863,0.0196078431372549}{#1}}%
\newcommand{\hleqformalargs}[1]{\textcolor[rgb]{0.690196078431373,0.250980392156863,0.0196078431372549}{#1}}%
\newcommand{\hlassignement}[1]{\textcolor[rgb]{0,0,0}{\textbf{#1}}}%
\newcommand{\hlpackage}[1]{\textcolor[rgb]{0.588235294117647,0.709803921568627,0.145098039215686}{#1}}%
\newcommand{\hlslot}[1]{\textit{#1}}%
\newcommand{\hlsymbol}[1]{\textcolor[rgb]{0,0,0}{#1}}%
\newcommand{\hlprompt}[1]{\textcolor[rgb]{0.2,0.2,0.2}{#1}}%

\usepackage{framed}
\makeatletter
\newenvironment{kframe}{%
 \def\at@end@of@kframe{}%
 \ifinner\ifhmode%
  \def\at@end@of@kframe{\end{minipage}}%
  \begin{minipage}{\columnwidth}%
 \fi\fi%
 \def\FrameCommand##1{\hskip\@totalleftmargin \hskip-\fboxsep
 \colorbox{shadecolor}{##1}\hskip-\fboxsep
     % There is no \\@totalrightmargin, so:
     \hskip-\linewidth \hskip-\@totalleftmargin \hskip\columnwidth}%
 \MakeFramed {\advance\hsize-\width
   \@totalleftmargin\z@ \linewidth\hsize
   \@setminipage}}%
 {\par\unskip\endMakeFramed%
 \at@end@of@kframe}
\makeatother

\definecolor{shadecolor}{rgb}{.97, .97, .97}
\definecolor{messagecolor}{rgb}{0, 0, 0}
\definecolor{warningcolor}{rgb}{1, 0, 1}
\definecolor{errorcolor}{rgb}{1, 0, 0}
\newenvironment{knitrout}{}{} % an empty environment to be redefined in TeX

\usepackage{alltt}


\usepackage{graphicx}
\usepackage{graphics}

\usepackage[english]{babel}
\usepackage[figuresright]{rotating}
\usepackage{bm}

\usepackage{enumerate}
\usepackage{float}
\usepackage{dsfont}

\usepackage{amsmath}
\usepackage{amsfonts}
\usepackage{amssymb}
%\usepackage{natbib}
\usepackage{authblk}

\usepackage{color}
%%%%%%%%%%%%%%%%%%  Author newcommand begin
\newtheorem{defi}{Definition}

\newcommand{\E}{\mathbb{E}}
\newcommand{\R}{\mathbb{R}}
\renewcommand{\P}{\mathbb{P}}
\newcommand{\Var}{\mathbb{V}ar}
\newcommand{\Cov}{\mathbb{C}ov}
\newcommand{\Comb}[2]{\left(\begin{array}{c}#1 \cr #2\end{array}\right) }
%%%%%%%%%%%%%%%%%  Author newcommend end

\usepackage[sc]{mathpazo}
\usepackage[T1]{fontenc}
\usepackage{geometry}
\geometry{verbose,tmargin=2.5cm,bmargin=2.5cm,lmargin=2.5cm,rmargin=2.5cm}
\setcounter{secnumdepth}{2}
\setcounter{tocdepth}{2}
\usepackage{url}
%\usepackage[unicode=true,pdfusetitle,
% bookmarks=true,bookmarksnumbered=true,bookmarksopen=true,bookmarksopenlevel=2,
% breaklinks=false,pdfborder={0 0 1},backref=false,colorlinks=false]
% {hyperref}
%\hypersetup{
% pdfstartview={XYZ null null 1}}
%\usepackage{breakurl}

\usepackage{courier}

%% Newcommand Utils
\newcommand{\RMQ}[1]{\par \hspace{-2cm}\textcolor{red}{RMQ : #1 }\par }
\newcommand{\ASS}[1]{\par \textcolor{blue}{ASSUMPTION : #1}\par}
\newcommand{\rcode}{\texttt{R}}


%% Newcommand for tuna report
\newcommand{\iscam}{\texttt{iSCAM}}
\newcommand{\admb}{\texttt{admb} }
\newcommand{\com}[1]{\textcolor{red}{#1}}


\newcommand{\fmsy}{F$_{\textnormal{MSY}}$}
\newcommand{\bmsy}{B$_{\textnormal{MSY}}$}


\newcommand{\Nt}{\boldsymbol N_t}


\title{Report for ICCAT-GBYP 04/2013}
\author[1]{Marie-Pierre Etienne\thanks{marie.etienne@agroparistech.fr}}
\author[2]{Tom Carruthers }
\author[2]{Murdoch McAlllister}
\affil[1]{AgroParisTech}
\affil[2]{UBC}
\IfFileExists{upquote.sty}{\usepackage{upquote}}{}

\begin{document}

\graphicspath{{/home/metienne/ICCAT/ICCAT-BFT/Report/}}
 






\maketitle
\section{Introduction}
\verb+admb+ and \verb+iSCAM+ are assumed to be installed, a \verb+PATH+ variable \verb+ADMB_HOME+ exists and contains the path to \verb+admb+ directory. In addition a \verb+PATH+ variable, called \verb+ISCAM+ contains the path to \verb+iscam+ executable.   
\section{R codes produced and how to use them}
All the codes and results described  or used in this report are freely
available on \url{https://github.com/MarieEtienne/ICCAT-BFT}. 

\iscam  has  been  developped  by  Steven  Martell,  is  available  on
\url{https://github.com/smartell/iSCAM} and described in \cite{Martell11}.

Using \iscam  required to write  some several datafiles as  inputs. To
produce an automatic process and in a context of reproducible science,
some \rcode codes  are used to translate a vpa  format file \verb+*.d1*+ in
inputs     file      for     \iscam.     The     main      file     is
\verb+sources/setISCAMFiles.R+  and required  the directory  where the
*.d1 file  is located. It  generates the input  files for \iscam  in a
directory based on the input  directory. With the current organisation
of the on GitHub, the input files are in \verb+ICCAT-BFT/Input/bfte/2012/...+, so
a directory  \verb+bfte/2012/..+ will  be created  in \verb+ICCAT-BFT+
and \iscam can belaunched from this directory.


To keep  the reproducibility of  this work,  the outputs showed  in this
report  are  automatically extracted  from  the  orginal R  files  and
outputs from \iscam.




\section{Description of iSCAM}
This section  sums up the principle  aspects of iSCAM used  on Bluefin
Tuna population dynamic data.
\subsection{Population model - Latent process Level}
iSCAM is an age structured model.  Fish are splitted into classes from
age $sage$ to year $nage$. Let us denote by $\Nt = (N_{1,t}, \ldots, N_{A,t})$ the number
of individuals in every classe age $a$ ($a\in [1,A]$) on year $t$ ($t\in [1;T]$).

\begin{gather}
  % \mbox{Initial state of the population}\\
  \mbox{Population dynamic after year syr (t>1)}\\
  N_{a,t}= \left\lbrace 
    \begin{array}{l}
      R_t, \quad a=1\\
      N_{a-1,t-1} \exp(-Z_{t-1, a-1}), \quad a\in [2;A-1]\\
      N_{a-1,t-1} \exp(-Z_{t-1, a-1}) + N_{a,t-1} \exp(-Z_{t-1, a}), \quad a\in [2;A-1],\\
    \end{array}  \right. \\
\end{gather}
 where $Z_{a,t}$  stands for the total  mortality rate at age  $a$ in
  year $y$ and $R_t$ is the recruitment at year $t$.

\RMQ{No variability in dynamic part of the model ?}

%\ASS{The first age class is 1 year, recruitment at year $t$ depends on
%  mature at year $t-1$, according to ICCAT report}
%\ASS{Beverton-Holt model is used for recruitment -- may be changed }

%\begin{gather}
%\mbox{Recruitment after year syr (t>1)}\\
%R_t= \frac{s_0 B_{t-1}}{1+\beta B_{t-1}} \exp{\epsilon^R_t}, \\
%\epsilon^R_t\underset{i.i.d}{\sim}\mathcal{N}\left(-\frac{\tau_R^2}{2}, \tau_R^2\right),\\
%\end{gather} 
\iscam is very similar to VPA from the recuitment point of view, since
it doesn't assume any stock recruitment relation ship. The recruitment
model is specified by the following equation
\begin{gather}
  R_t =\bar{R}e^{\omega_t}
\end{gather}

\begin{gather}
\mbox{Definition of mature biomass}\\
B_t= \sum_{a=1}^A N_{t,a} f_a,
\end{gather} 
$f_a$ being the fertility at age $a$. 

\ASS{$f_a$ doesn't depend  on year $t$, fertility at  age is considred
  as fixed over year}

  
  
\begin{gather}
  \mbox{Definition of total mortality rate ate age}\\
  Z_{t,a}= M_t + \sum_{k=1}^K F_{k,t} v_{k, t,a}, \quad \mbox{gear} k,
  \mbox{class }a, \mbox{year }t.   \\
  M_t =M_{t-1}\ \exp{\phi_t},\quad \phi_t \underset{i.i.d}{\sim}\mathcal{N}\left(0, \tau_{\phi}^2\right)
\end{gather} 
\ASS{To account  for the commercial  catch (only one time  series and
for the seven abundance indices,  eight gears have been declared. gear
one corresponds  to commercial fisheries,  gear 2 to 8  corresponds to
the seven abundance indices, therefore $F_{f,t}=0$ for $k \geq 2$}

\iscam    allows     to    specify    different    form     for    the
selectivity/vulnerability.  When  age composition data  are available,
choice has been made to model the selectivity using Bspline curves.
\begin{knitrout}
\definecolor{shadecolor}{rgb}{0.969, 0.969, 0.969}\color{fgcolor}\begin{kframe}
\begin{verbatim}
## ## SELECTIVITY PARAMETERS Columns for gear                                   ##
## ## OPTIONS FOR SELECTIVITY (isel_type):                                      ##
## 3 1 1  # 1  -selectivity type ivector(isel_type) for gear
\end{verbatim}
\end{kframe}
\end{knitrout}


The numbers  of nodes for the  Bspline curves can be  chosen equal for
all gear with age composition data 

\begin{knitrout}
\definecolor{shadecolor}{rgb}{0.969, 0.969, 0.969}\color{fgcolor}\begin{kframe}
\begin{verbatim}
## 5 0 0  # 4  -No. of age nodes for each gear (0=ignore)
\end{verbatim}
\end{kframe}
\end{knitrout}


For gear  4 corresponding to the  Norvegian Purse seine index,  no age
composition data  are available.  In \cite{tuna2012}, it  is indicated
that this index is  relevant only for the last class  age. Since it is
not  directly   possible  to  specify   a  selectivity  with   only  0
coeffeicient except for one age,  a logistic selectivity curve has been
adopted with the following parameters.
\begin{knitrout}
\definecolor{shadecolor}{rgb}{0.969, 0.969, 0.969}\color{fgcolor}\begin{kframe}
\begin{verbatim}
## 6 6 6  # 2  -Age/length at 50% selectivity (logistic)
## 1 1 1  # 3  -STD at 50% selectivity (logistic)
\end{verbatim}
\end{kframe}
\end{knitrout}


\subsection{Observation Level}
\subsection{Age composition data}

Following  \cite{Schnute+95}, \iscam  uses by  default a  multivariate
logistic  function  for  age  composition data.  It  is  assumed  that
$p_{atk}$ which is the  proportion of fish of age $a$  in year $t$ for
gear $k$ is drawn from the following distribution (gear $k$ is omitted
for clarity) which is defined thanks to a latent variable $X_{at}$
\begin{gather}
\epsilon^{C}_{at} \overset{i.i.d}{\sim} \mathcal{N}(0,\tau^2_C)\\
X_{at}       =       \log{\mu_{at}}       +       \epsilon_{at}       -
\frac{1}{A}\sum_{a=1}^A\left(\log{\mu_{at}} + \epsilon_{at} \right) \\
e^{X_{at}} = \frac{\mu_{at}e^{\epsilon_{at}}}{ \left(\prod_{a=1}^A \mu_{at}e^{\epsilon_{at}}\right) ^{1/A}}\\
p_{at} = \frac{e^{X_{at}} } {\sum_{a=1}^A e^{X_{at}} }\\
\end{gather}
\subsection{Abundance indices}
The total vulnerable biomass at year $t$ for gear $k$ is definde by
\begin{gather}
V_{k,t}=\sum_{a=1}^A N_{t,a} e^{-\lambda_{k,t} Z_{t,a}} v_{k,a} w_a,\\
\end{gather}
 where $\lambda_{k,t}$ is the fraction  of the mortality to adjust for
 survey  timing,  it  is  specified  by the  user,  they  have  v=been
 specified according to the input data file 
\begin{knitrout}
\definecolor{shadecolor}{rgb}{0.969, 0.969, 0.969}\color{fgcolor}\begin{kframe}
\begin{verbatim}
## ## Survey timing (if 0, the gear has no associated index)
## 0	 0.5	0.5
\end{verbatim}
\end{kframe}
\end{knitrout}

 
\begin{gather}
I_{k,t} = q_k V_{k,t} \exp{\epsilon_{k,t}^I}\\
\epsilon_{k,t}^I \overset{i.i.d}{\sim} \mathcal{N}(0,\tau_I^2)\\
\end{gather}
\subsection{Catch}
Catch  with  gear  $k$  in  year  $t$  is  denoted  by  $C_{k,t}$  and
definedusing Baranov Catch equation by
\begin{gather}
C_{k,t}   =   \sum_{a=1}^A   \frac{N_{t,a}   w_a   F_{k,t}   v_{k,t,a}
  (1-e^{-Z_{t,a}}) }{Z_{t,a}},\\
\hat{C}_{k,t} = C_{k,t} \exp{\epsilon_{k,t}^C}, \quad \epsilon_{k,t}\overset{i.i.d}{\sim}\mathcal{N}(0, \tau^2_C)\\
\end{gather}



\section{Fitting iSCAM on Bluefin Tuna Data}

\subsection{Description of the data set vailable data}

\subsubsection{Abundance indices}
Seven abundance indices are available 
\begin{knitrout}
\definecolor{shadecolor}{rgb}{0.969, 0.969, 0.969}\color{fgcolor}

{\centering \includegraphics[width=\maxwidth]{figure/ICCAT-Abundance} 

}



\end{knitrout}

\subsection{Available catch}
\begin{knitrout}
\definecolor{shadecolor}{rgb}{0.969, 0.969, 0.969}\color{fgcolor}

{\centering \includegraphics[width=\maxwidth]{figure/ICCAT-Catch} 

}



\end{knitrout}

\subsubsection{Data on Selectivity}

\begin{knitrout}
\definecolor{shadecolor}{rgb}{0.969, 0.969, 0.969}\color{fgcolor}

{\centering \includegraphics[width=\maxwidth]{figure/ICCAT-Selectivity} 

}



\end{knitrout}

   
\subsection{Fixed parameters}
The mortality at age is fixed to the values of the VPA approach.
\begin{knitrout}
\definecolor{shadecolor}{rgb}{0.969, 0.969, 0.969}\color{fgcolor}\begin{kframe}
\begin{verbatim}
##  [1] 0.490 0.240 0.240 0.240 0.240 0.200 0.175 0.150 0.125 0.100
\end{verbatim}
\end{kframe}
\end{knitrout}


$$\Phi = (l_\infty, k, t_o,a,b,\dot{a},\dot{\gamma})$$
\begin{knitrout}
\definecolor{shadecolor}{rgb}{0.969, 0.969, 0.969}\color{fgcolor}\begin{kframe}
\begin{verbatim}
## linf  =  319
## k  =  0.093
## to  =  -0.97
##  sclw = 1.95e-05 #1.95e-5 #scaler in length-weight allometry
## plw =  3.009  #power in length-weight allometry
## m50 =  4 #50% maturity
## std50 =  0.8 #std at 50% maturity
\end{verbatim}
\end{kframe}
\end{knitrout}




\subsection{Initial values for estiamted parameters}
$$\theta   =  (R_0,   \kappa,   M,  \bar{R},   \rho,  v^2,   \gamma_k,
\boldsymbol{F}_{t}, (phi_t)_t=1^T, (\epsilon_t^R)_{t=1}^T)$$


	%% ro          = mfexp(theta(1));
	%% dvariable h = theta(2);
	%% m           = mfexp(theta(3));
	%% log_avgrec  = theta(4);
	%% log_recinit = theta(5);
	%% rho         = theta(6);
	%% varphi      = sqrt(1.0/theta(7));
	%% sig         = sqrt(rho) * varphi;
	%% tau         = sqrt(1-rho) * varphi;

        
        The  control  file  provides  initial  values  for  parameters
        $\theta$, the  following values  have been used  to initialise
        the model.

\subsubsection*{Value for $R_0$}


\subsubsection*{Value for $\bar{R}$}
If the  model doesn't supposed  unfished conditions at  starting year,
the recruitment at  the first year is  not $R_0$. It is  assumed to be
equal to
$$R_{t=1} = \bar{R}_{init} \exp{\epsilon_1^R}$$
In \iscam code, the recruitmnent at year $t$ is defined by
$$R_t(t)=\bar{R} \exp{\epsilon_t^R}, $$
  where $\epsilon_t^R\overset{i.i.d}{\sim} \mathcal{N}(0,\sigma_R^2)$.


\subsection{Assumption on vulnerability parameters}
When catch at age data are available, a vunlnerability curve using cubic Bsplines is fitted and currently it is assumed to be constant over time. (\com{Option 3 in selectivity option for \iscam})
It is currently not possible with \iscam to specify 0 for vulnerability, since it is expressed in log scale and because \admb behaves better with differentiable functions.



\section{First runs on one data file}

\begin{knitrout}
\definecolor{shadecolor}{rgb}{0.969, 0.969, 0.969}\color{fgcolor}\begin{kframe}


{\ttfamily\noindent\itshape\color{messagecolor}{\#\# Loading required package: MASS}}

{\ttfamily\noindent\itshape\color{messagecolor}{\#\# \\\#\# Attaching package: 'MASS'}}

{\ttfamily\noindent\itshape\color{messagecolor}{\#\# The following object(s) are masked \_by\_ '.GlobalEnv':\\\#\# \\\#\#\ \ \ \  survey}}

{\ttfamily\noindent\itshape\color{messagecolor}{\#\# Loading required package: KernSmooth}}

{\ttfamily\noindent\itshape\color{messagecolor}{\#\# KernSmooth 2.23 loaded\\\#\# Copyright M. P. Wand 1997-2009}}

{\ttfamily\noindent\color{warningcolor}{\#\# Warning: NAs introduced by coercion}}\end{kframe}
\end{knitrout}


The file used for this first run is 
\begin{knitrout}
\definecolor{shadecolor}{rgb}{0.969, 0.969, 0.969}\color{fgcolor}\begin{kframe}
\begin{verbatim}
## /home/metienne/ICCAT/ICCAT-BFT/Inputs/bfte/2012/vpa/reported/low/bfte2012.d1
\end{verbatim}
\end{kframe}

{\centering \includegraphics[width=\maxwidth]{figure/ICCAT-Selectivity2} 

}



\end{knitrout}

\subsection{Summary of obtained results}
\begin{knitrout}
\definecolor{shadecolor}{rgb}{0.969, 0.969, 0.969}\color{fgcolor}\begin{kframe}
\begin{alltt}
\hlcomment{# print(res$A) print(res$Ahat) print(res$A_nu)}
\end{alltt}
\end{kframe}
\end{knitrout}

\begin{knitrout}
\definecolor{shadecolor}{rgb}{0.969, 0.969, 0.969}\color{fgcolor}\begin{kframe}
\begin{alltt}
\hlfunctioncall{print}(res$fmsy)
\end{alltt}
\begin{verbatim}
## [1] 0.09455
\end{verbatim}
\begin{alltt}
\hlfunctioncall{print}(res$msy)
\end{alltt}
\begin{verbatim}
## [1] 83363400
\end{verbatim}
\begin{alltt}
\hlfunctioncall{print}(res$bmsy)
\end{alltt}
\begin{verbatim}
## [1] 655735000
\end{verbatim}
\begin{alltt}
\hlfunctioncall{print}(res$bo)
\end{alltt}
\begin{verbatim}
## [1] 1.925e+09
\end{verbatim}
\begin{alltt}
\hlfunctioncall{print}(res$ro)
\end{alltt}
\begin{verbatim}
## [1] 7328580
\end{verbatim}
\begin{alltt}
\hlfunctioncall{print}(res$q)
\end{alltt}
\begin{verbatim}
## [1] 1.127 1.335
\end{verbatim}
\end{kframe}
\end{knitrout}


\begin{knitrout}
\definecolor{shadecolor}{rgb}{0.969, 0.969, 0.969}\color{fgcolor}\begin{kframe}
\begin{alltt}
\hlfunctioncall{print}(res$steepness)
\end{alltt}
\begin{verbatim}
## [1] 0.6974
\end{verbatim}
\end{kframe}
\end{knitrout}


\subsection{Fishing effort}
\begin{knitrout}
\definecolor{shadecolor}{rgb}{0.969, 0.969, 0.969}\color{fgcolor}\begin{kframe}
\begin{alltt}
\hlfunctioncall{plot}(x = res$yr, y = res$ft[1, ], xlab = \hlstring{"Year"}, ylab = \hlstring{"Fishing effort"}, type = \hlstring{"b"})
\end{alltt}
\end{kframe}

{\centering \includegraphics[width=\maxwidth]{figure/ICCAT-Ft} 

}



\end{knitrout}


\subsection{Spawning biomass}
\begin{knitrout}
\definecolor{shadecolor}{rgb}{0.969, 0.969, 0.969}\color{fgcolor}\begin{kframe}
\begin{alltt}
\hlfunctioncall{plot}(x = res$yrs, y = res$sbt/1000, xlab = \hlstring{"Year"}, ylab = \hlstring{"Spawning \hlfunctioncall{biomass} (tons)"}, 
    type = \hlstring{"b"})
\end{alltt}
\end{kframe}

{\centering \includegraphics[width=\maxwidth]{figure/ICCAT-SBT} 

}



\end{knitrout}


\subsection{Kobe Plot}
\begin{knitrout}
\definecolor{shadecolor}{rgb}{0.969, 0.969, 0.969}\color{fgcolor}\begin{kframe}
\begin{alltt}
\hlfunctioncall{plot}(res$Fstatus[1, ] ~ res$Bstatus[1:62], xlab = \hlstring{"B/Bmsy"}, type = \hlstring{"l"}, ylab = \hlstring{"F/Fmsy"}, 
    xlim = \hlfunctioncall{c}(0, 2), ylim = \hlfunctioncall{c}(0, 2))
\end{alltt}
\end{kframe}

{\centering \includegraphics[width=\maxwidth]{figure/ICCAT-KobePlot} 

}



\end{knitrout}

\begin{knitrout}
\definecolor{shadecolor}{rgb}{0.969, 0.969, 0.969}\color{fgcolor}\begin{kframe}
\begin{alltt}
\hlfunctioncall{detach}(Info)
\end{alltt}
\end{kframe}
\end{knitrout}


\section{Results coming from VPA}
\section{Simulation tests}
\section{Stock assesment methodology}
\bibliographystyle{apalike}
\bibliography{biblio}

\end{document} 
