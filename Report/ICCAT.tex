\documentclass[a4paper]{article}\usepackage{graphicx, color}
%% maxwidth is the original width if it is less than linewidth
%% otherwise use linewidth (to make sure the graphics do not exceed the margin)
\makeatletter
\def\maxwidth{ %
  \ifdim\Gin@nat@width>\linewidth
    \linewidth
  \else
    \Gin@nat@width
  \fi
}
\makeatother

\definecolor{fgcolor}{rgb}{0.2, 0.2, 0.2}
\newcommand{\hlnumber}[1]{\textcolor[rgb]{0,0,0}{#1}}%
\newcommand{\hlfunctioncall}[1]{\textcolor[rgb]{0.501960784313725,0,0.329411764705882}{\textbf{#1}}}%
\newcommand{\hlstring}[1]{\textcolor[rgb]{0.6,0.6,1}{#1}}%
\newcommand{\hlkeyword}[1]{\textcolor[rgb]{0,0,0}{\textbf{#1}}}%
\newcommand{\hlargument}[1]{\textcolor[rgb]{0.690196078431373,0.250980392156863,0.0196078431372549}{#1}}%
\newcommand{\hlcomment}[1]{\textcolor[rgb]{0.180392156862745,0.6,0.341176470588235}{#1}}%
\newcommand{\hlroxygencomment}[1]{\textcolor[rgb]{0.43921568627451,0.47843137254902,0.701960784313725}{#1}}%
\newcommand{\hlformalargs}[1]{\textcolor[rgb]{0.690196078431373,0.250980392156863,0.0196078431372549}{#1}}%
\newcommand{\hleqformalargs}[1]{\textcolor[rgb]{0.690196078431373,0.250980392156863,0.0196078431372549}{#1}}%
\newcommand{\hlassignement}[1]{\textcolor[rgb]{0,0,0}{\textbf{#1}}}%
\newcommand{\hlpackage}[1]{\textcolor[rgb]{0.588235294117647,0.709803921568627,0.145098039215686}{#1}}%
\newcommand{\hlslot}[1]{\textit{#1}}%
\newcommand{\hlsymbol}[1]{\textcolor[rgb]{0,0,0}{#1}}%
\newcommand{\hlprompt}[1]{\textcolor[rgb]{0.2,0.2,0.2}{#1}}%



\usepackage{framed}
\makeatletter
\newenvironment{kframe}{%
 \def\at@end@of@kframe{}%
 \ifinner\ifhmode%
  \def\at@end@of@kframe{\end{minipage}}%
  \begin{minipage}{\columnwidth}%
 \fi\fi%
 \def\FrameCommand##1{\hskip\@totalleftmargin \hskip-\fboxsep
 \colorbox{shadecolor}{##1}\hskip-\fboxsep
     % There is no \\@totalrightmargin, so:
     \hskip-\linewidth \hskip-\@totalleftmargin \hskip\columnwidth}%
 \MakeFramed {\advance\hsize-\width
   \@totalleftmargin\z@ \linewidth\hsize
   \@setminipage}}%
 {\par\unskip\endMakeFramed%
 \at@end@of@kframe}
\makeatother

\definecolor{shadecolor}{rgb}{.97, .97, .97}
\definecolor{messagecolor}{rgb}{0, 0, 0}
\definecolor{warningcolor}{rgb}{1, 0, 1}
\definecolor{errorcolor}{rgb}{1, 0, 0}

\usepackage{alltt}


\usepackage{graphicx}
\usepackage{graphics}
\usepackage{subcaption}

\usepackage[english]{babel}
\usepackage[figuresright]{rotating}
\usepackage{bm}

\usepackage{enumerate}
\usepackage{float}
\usepackage{dsfont}

\usepackage{amsmath, amsfonts,amssymb}
%\usepackage{natbib}
\usepackage{authblk}

\usepackage{color}
%%%%%%%%%%%%%%%%%%  Author newcommand begin
\newtheorem{defi}{Definition}

\newcommand{\E}{\mathbb{E}}
\newcommand{\R}{\mathbb{R}}
\renewcommand{\P}{\mathbb{P}}
\newcommand{\Var}{\mathbb{V}ar}
\newcommand{\Cov}{\mathbb{C}ov}
\newcommand{\Comb}[2]{\left(\begin{array}{c}#1 \cr #2\end{array}\right) }
%%%%%%%%%%%%%%%%%  Author newcommend end

\usepackage[sc]{mathpazo}
\usepackage[T1]{fontenc}
\usepackage{geometry}
\geometry{verbose,tmargin=2.5cm,bmargin=2.5cm,lmargin=2.5cm,rmargin=2.5cm}
\setcounter{secnumdepth}{2}
\setcounter{tocdepth}{2}
\usepackage{url}


\usepackage{courier}

%% Newcommand Utils
\newcommand{\RMQ}[1]{\par \hspace{-2cm}\textcolor{red}{RMQ : #1 }\par }
\newcommand{\ASS}[1]{\par \textcolor{blue}{ASSUMPTION : #1}\par}
\newcommand{\rcode}{\texttt{R}}


%% Newcommand for tuna report
\newcommand{\iscam}{\texttt{iSCAM }}
\newcommand{\admb}{\texttt{admb} }
\newcommand{\com}[1]{\textcolor{red}{#1}}


\newcommand{\fmsy}{F$_{\textnormal{MSY}}$}
\newcommand{\bmsy}{B$_{\textnormal{MSY}}$}


\newcommand{\Nt}{\boldsymbol N_t}

\newcounter{resultz}
\newenvironment{resultz}{\refstepcounter{resultz}\equation}{\tag{R\theresultz}\endequation}

\title{Report for ICCAT-GBYP 04/2013}
\author[1]{Marie-Pierre Etienne\thanks{marie.etienne@agroparistech.fr}}
\author[2]{Tom Carruthers }
\author[2]{Murdoch McAlllister}
\affil[1]{AgroParisTech}
\affil[2]{UBC}
\IfFileExists{upquote.sty}{\usepackage{upquote}}{}

\begin{document}

\graphicspath{{/home/metienne/ICCAT/ICCAT-BFT/Report/figure}}
 






\maketitle
\tableofcontents
\clearpage

\section{Introduction}

The work reported in this document aims at 
\begin{enumerate}
 \item examining the opportunity of using a statistical catch at age for Bluefin tuna stock evaluation,
 \item higlighhting the structure and the assumptions of the chosen statistical model (\iscam developped by S. Martell \cite{Martell12})
 \item running a simulation study for evaluating the quality of the estimation procedure,
  \item producing operationnal codes for running a stock evaluation on Bluefin Tuna east stock with a statistical catch at age model,
  \item comparing the results with the results presented in previous stock assesment
  \end{enumerate}
  
The report is split in five main sections. The first section describes the model proposed in \iscam. The second section described briefly the data used  for this test.
Section three presents the practical choice made to run \iscam on this dataset. The simulation study is presented in section \ref{sec:simulation}.
The complete results are developped in the last main section. The discussion sums up the main result and highlights the strength and the weakness of \iscam.
All the codes used to produce this work are freely available on \url{https://github.com/MarieEtienne/ICCAT-BFT}

\section{Description of the main aspects of \iscam }
This section  sums up the principle  aspects of iSCAM used  on Bluefin
Tuna population dynamic data.
\subsection{Latent process Level}
\subsubsection{Population model}
\iscam is an age structured model.  Fish population is splitted into age classes from
age $sage$ to age $nage$. Let us denote by $\Nt = (N_{1,t}, \ldots, N_{A,t})$ the number
of individuals in every classe age $a$ ($a\in [1,A]$) on year $t$ ($t\in [1;T]$).

\begin{gather}
  % \mbox{Initial state of the population}\\
  \mbox{Population dynamic after year syr (t>1)}\\
  N_{a,t}= \left\lbrace 
    \begin{array}{l}
      R_t, \quad a=1\\
      N_{a-1,t-1} \exp(-Z_{t-1, a-1}), \quad a\in [2;A-1]\\
      N_{a-1,t-1} \exp(-Z_{t-1, a-1}) + N_{a,t-1} \exp(-Z_{t-1, a}), \quad a\in [2;A-1],\\
    \end{array}  \right. \\
\end{gather}
 where $Z_{a,t}$  stands for the total  mortality rate at age  $a$ in
  year $y$ and $R_t$ is the recruitment at year $t$.




\begin{gather}
\mbox{Recruitment after year syr (t>1)}\\
R_t= \frac{s_0 B_{t-1}}{1+\beta B_{t-1}} \exp{\epsilon^R_t}, \\
\epsilon^R_t\underset{i.i.d}{\sim}\mathcal{N}\left(-\frac{\tau_R^2}{2}, \tau_R^2\right),\\
\end{gather} 

\iscam assumes a stock recruitment relationship, Beverton and Holt
(BH) or Ricker (R) model may be used which links the mature biomass at
time $t$ $B_t$and the recruitment at time $t+1$. The presented work used the BH model. 
There is slight trick in \iscam, since this relation ship is estimated
in the latest phase of the estimation process. During the first phases the recruitment
model is specified by the following equation
\begin{gather}
\label{eq:iscamRec}  R_t =\bar{R}e^{\omega_t},
\end{gather}
where  $\omega_t\overset{i.i.d}{\sim}  \mathcal{N}(0,  \tau^2_R)$  and
$\bar{R}$ is the average recruitment. This
approach  of the  early  phases is  very  close to  the  VPA from  the
recruitment  point  of  view,  since   it  doesn't  assume  any  stock
recruitment relationship. 



\begin{gather}
\mbox{Definition of mature biomass}\\
B_t= \sum_{a=1}^A N_{t,a} f_a,
\end{gather} 
$f_a$ being the fertility at age $a$. 

$f_a$ doesn't depend  on year $t$, fertility at  age is considred
  as fixed over year

  
  
\begin{gather}
  \mbox{Definition of total mortality rate ate age}\\
  Z_{t,a}= M_a + \sum_{k=1}^K F_{k,t} v_{k, t,a}, \quad \mbox{gear }k,
  \mbox{class }a, \mbox{year }t.   \\
\end{gather} 
In  the  BFT  data  available  there is  only  one  time  series  for
 catch (catch are  not split among different  gears), considering that
 the gear index corresponding to the 
commercial fisheries equals one, the previous expression may be simplified as~:

\begin{gather}
  \mbox{Definition of total mortality rate ate age}\\
  Z_{t,a}= M_a + F_t v_{1, t,a}, \quad \mbox{gear } 1,
  \mbox{class }a, \mbox{year }t,   \\
\end{gather} 

where $F_t$ is the instantaneous  fishing mortality and $v_{1,t,a}$ is
the vulnerability for gear $1$, in year $t$ for age classe a.

\subsubsection{Vulnerability model}
\label{sec:vulnerability}
\iscam    allows     to    specify    different    form     for    the
selectivity/vulnerability. The vulnerability  may either be completely
specified by the user, or when  age composition data are available the
vulnerability may be  inferred. In the latter case,  different form of
selectvity curve may be specified:  for example the selectivity may be
chosen as a logistic function of age, or even, with a more flexible approach, using B-spline.  The B-splines
functions may  even model a  variation accross years. In  this work,
only logistic or simple Bspline has been used to circumvent estimation issues.


\subsection{Observation Level}
\subsubsection{Age composition data}

Following  \cite{Schnute+95}, \iscam  uses by  default a  multivariate
logistic  function  for  age  composition data.  It  is  assumed  that
$p_{atk}$ which is the  proportion of fish of age $a$  in year $t$ for
gear $k$ is drawn from the following distribution (gear $k$ is omitted
for clarity) which is defined thanks to a latent variable $X_{at}$
\begin{gather}
\epsilon^{A}_{at} \overset{i.i.d}{\sim} \mathcal{N}(0,\tau^2_A)\nonumber \\ 
X_{at}       =       \log{\mu_{at}}       +       \epsilon_{at}^A       -
\frac{1}{A}\sum_{a=1}^A\left(\log{\mu_{at}} + \epsilon_{at}^A \right) \nonumber \\
e^{X_{at}} = \frac{\mu_{at}e^{\epsilon_{at}^A}}{ \left(\prod_{a=1}^A \mu_{at}e^{\epsilon_{at}^A}\right) ^{1/A}}\nonumber\\
p_{at} = \frac{e^{X_{at}} } {\sum_{a=1}^A e^{X_{at}} }\\
\end{gather}
The multivariate logistic distribution  avoids the drawback of a very high
precision when using a classical multinomiale distribution.

\subsubsection{Abundance indices}
The total vulnerable biomass at year $t$ for gear $k$ is defined by
\begin{gather}
V_{k,t}=\sum_{a=1}^A N_{t,a} e^{-\lambda_{k,t} Z_{t,a}} v_{k,a} w_a,
\end{gather}
 where $\lambda_{k,t}$ is the fraction  of the mortality to adjust for
 survey timing, it is specified by the user.

 
\begin{gather}
I_{k,t} = q_k V_{k,t} \exp{\epsilon_{k,t}^I}\\
\epsilon_{k,t}^I \overset{i.i.d}{\sim} \mathcal{N}(0,\tau_I^2)\nonumber
\end{gather}
\subsubsection{Catch}
Catch  with  gear  $k$  in  year  $t$  is  denoted  by  $C_{k,t}$  and
defined using Baranov Catch equations by
\begin{gather}
\hat{C}_{k,t}   =   \sum_{a=1}^A   \frac{N_{t,a}   w_a   F_{k,t}   v_{k,t,a}
  (1-e^{-Z_{t,a}}) }{Z_{t,a}},\\
C_{k,t} = \hat{C}_{k,t} \exp{\epsilon_{k,t}^C}, \quad \epsilon_{k,t}\overset{i.i.d}{\sim}\mathcal{N}(0, \tau^2_C)\nonumber
\end{gather}

\subsection{Summary of the symbols in \iscam}

Table  \ref{tab:parameters}  and  \ref{tab:symbols}  recall  the  main
symbol used in this report and give their corresponding name in \iscam
description available  in \cite{Martell11}. Some of  the names have
been changed to unify the notation  standards. When the name is commun
to this paper and \iscam users Guide, nothing is specified.


\begin{table}[ht]
\centering
\begin{tabular}{p{3cm}p{4cm}p{8cm}}
  \hline
Parameters & Name in \iscam & Signification \\ 
  \hline
$\tau_C$& $\sigma_C$ & Standard deviation in observed catch\\
$\tau_A$ & $\tau$,  mentionned only through its estimation  on page 21
of \cite{Martell12} & standard deviation in the multivariate logistic function \\
$v_{k,a}$& &vulnerability for gear $k$ at age $a$\\
$R_0$ && Recruitment in unfished conditions\\
$h$ && Steepness\\
$q_k$ & & Catchability coefficient for gear $k$\\
$\rho$ & & the proportion of total variation allocated to the variance
of the survey index\\
$\varphi^2$ & $1/\mathcal{V}^2$& total variance $\varphi^2=\tau^2_R+\tau^2_I$\\
$\tau_I=\varphi*\sqrt{\rho}$ & $\sigma$ & Standard deviation in the survey index\\
$\tau_R=\varphi*\sqrt{1-\rho}$ & $\tau$& Standard deviation in process
errors\\
   \hline
\end{tabular}
\caption{Leading parameters}
\label{tab:parameters}
\end{table}

\subsubsection{Other symbols}

\begin{table}[ht]
\centering
\begin{tabular}{p{3.5cm}p{3.5cm}p{8cm}}
  \hline
Name & Name in \iscam & Signification \\ 
  \hline
$A$ & & last class age\\
$\kappa=4h / (1-h)$ & & Good year compensation ratio for BH\\
$s_0= \kappa/\Phi_E$ & & BH relationship\\
$\beta=(\kappa-1) / (R_0 \Phi_E)$ &&BH relationship\\
$\epsilon_t^R$ & $\delta_t$& recruitment errors\\
$\epsilon_{t,a}^A$ & $\eta_{t,a}$& errors in the multivariate logistic \\
$\epsilon_{k,t}^I$ & $\epsilon_{k,t}$& residuals in abundance survey\\
$\epsilon_t^C$ & $\eta_{k,t}$& residuals in catch data\\
$\bar{R}$ &  &Average recruitment  used in  firt estimation  phases as
mentionned in equation \ref{eq:iscamRec}\\
$B_t$ & &Spawning biomass in numbers \\
$N_{a,t}$&& Number of individuals of age a in year t\\
$f_a$ && fecundity at age $a$\\
$M_a$ && instantaneous mortality at age $a$\\
$F_{k,t}, F_{t}$&& Instantaneous fishing mortality  for gear k in year
t, since  there is only one  commercial gear considered, index  $k$ is
mostly omitted\\
$Z_{a,t}$& &Instantaneous total mortality for age a in year t\\ 
$X_{a,t}$&   not  named   in  \iscam&   Quantities  involved   in  the
multivariate logistic\\
$\mu_{a,t}$ &  & unormalized proportion at age\\
& $\widehat{p_{a,t}}=\mu_{a,t}/\sum_a \mu_{a,t}$& proportion at age\\
$p_{a,t}$& & Observed proprotion at age\\
$V_{k,t}$& & Total vulnerable biomass for gear k in year t\\
${w}_{a}$& &Weight at age a\\
   \hline
\end{tabular}
\caption{List of symbols used in \iscam}
\label{tab:symbols}
\end{table}


\clearpage\clearpage

\section{Description of Bluefin Tuna data}
The presented  work ocus on  Bluefin Tuna  east stock data.  There are
seven abundance indices and the total commercial catch available. 





\subsection{Abundance indices}

Seven abundance indices are available.   the oldest one starts in year
1950,  the last  indices considered  stop in  2011. Three  indices are
assumed to be proportional to  the total number of tuna (\verb+SM_TP+,
\verb+LL_JP1+,  \verb+LL_JP2+),  the four  others  are  assumed to  be
proportionnal to the total biomass (expressed as weight).

\begin{figure}
{\centering \includegraphics[width=0.7\maxwidth]{figure/ICCAT-Abundance} 
}
\caption{Scaled abundance indices }
\end{figure}




\subsection{Available catch}
The  total catch  in  numbers  are given  from  1950  to 2011.  Figure
\ref{fig:TotCatch} illsutrates the evolution of the ctach across years.
\begin{figure}
\centering \includegraphics[width=0.7\maxwidth]{figure/ICCAT-Catch} 
\caption{Catch  splitted by  age over  the whole  period. Area  of red
  circles is proportionnal to the Catch}
\label{fig:TotCatch}
\end{figure}


\subsection{Data on Selectivity}
Catch at age  data are available for the commercial  fisheries and for
six of the seven abundance  indices. It is specified that \verb+NW_PS+
is proportionnal to last age class (more than 10 years old tuna).




To investigate the  change in selectivity, we can  look at composition
of the catch per gear and per year
\begin{figure}
{\centering \includegraphics[width=\maxwidth]{figure/ICCAT-Selectivity11} 
}
\caption{Catch splitted by age and gear over the whole period. Area of the
  circles is proportionnal to the Catch}
\end{figure}






\clearpage
\section{Practical aspects of running \iscam on Bluefin Tuna data}

\subsection{Tunning up \iscam - choices}
%\ASS{The first age class is 1 year, recruitment at year $t$ depends on
%  mature at year $t-1$, according to ICCAT report}
%\ASS{Beverton-Holt model is used for recruitment -- may be changed }

\ASS{To account  for the commercial  catch (only one time  series and
for the seven abundance indices),  eight gears have been declared. gear
one corresponds  to commercial fisheries,  gear 2 to 8  corresponds to
the seven abundance indices, therefore $F_{f,t}=0$ for $k \geq 2$}

\paragraph{Mortality}
The mortality at age is fixed to the values used for the VPA approach.

\begin{table}[ht]
\centering
\begin{tabular}{rrrrrrrrrrr}
  \hline
 & 1 & 2 & 3 & 4 & 5 & 6 & 7 & 8 & 9 & 10 \\ 
  \hline
Mortality & 0.49 & 0.24 & 0.24 & 0.24 & 0.24 & 0.20 & 0.17 & 0.15 & 0.12 & 0.10 \\ 
   \hline
\end{tabular}
\end{table}


\paragraph{Selectivity}
As mentionned in \iscam allows to specify different form for the
selectivity/vulnerability.  When  age composition data  are available,
choice has been made to model  the selectivity using Bspline curves or
logistic function.



% ## ## SELECTIVITY PARAMETERS Columns for gear                                   ##
% ## ## OPTIONS FOR SELECTIVITY (isel_type):                                      ##
% ## 3 1 1 6 1 3 13  # 1  -selectivity type ivector(isel_type) for gear



The numbers  of nodes for the  Bspline curves can be  chosen equal for
all gear with age composition data 

% \begin{knitrout}
% \definecolor{shadecolor}{rgb}{0.969, 0.969, 0.969}\color{fgcolor}\begin{kframe}
% \begin{verbatim}
% ## 5 0 0 0 0 5 0  # 4  -No. of age nodes for each gear (0=ignore)
% \end{verbatim}
% \end{kframe}
% \end{knitrout}

For gear  4 corresponding to the  Norvegian Purse seine index,  no age
composition data  are available.  In \cite{tuna2012}, it  is indicated
that this index is  relevant only for the last class  age. Since it is
not  directly   possible  to  specify   a  selectivity  with   only  0
coefficient except for one age in \iscam, a logistic selectivity curve has been
adopted with the following parameters.
% \begin{knitrout}
% \definecolor{shadecolor}{rgb}{0.969, 0.969, 0.969}\color{fgcolor}\begin{kframe}
% \begin{verbatim}
% ## 6 6 6 9.9 6 6 6  # 2  -Age/length at 50% selectivity (logistic)
% ## 1 1 1 0.1 1 1 1  # 3  -STD at 50% selectivity (logistic)
% \end{verbatim}
% \end{kframe}
% \end{knitrout}


\paragraph{Abundance indices}



The total vulnerable biomass at year $t$ for gear $k$ is defined by
\begin{gather}
V_{k,t}=\sum_{a=1}^A N_{t,a} e^{-\lambda_{k,t} Z_{t,a}} v_{k,a} w_a,
\end{gather}
 where $\lambda_{k,t}$ is the fraction  of the mortality to adjust for
 survey timing, it is specified by the user. They have been
 specified according to the input data file 
% \begin{knitrout}
% \definecolor{shadecolor}{rgb}{0.969, 0.969, 0.969}\color{fgcolor}\begin{kframe}
% \begin{verbatim}
% ## ## Survey timing (if 0, the gear has no associated index)
% ## 0	 0.5	0.5	0.5	0.0833333333333333	0.5	0.5
% \end{verbatim}
% \end{kframe}
% \end{knitrout}


\paragraph{Recruitment}


\paragraph{Life trait specification}

$$\Phi = (l_\infty, k, t_o,a,b,\dot{a},\dot{\gamma})$$
\begin{verbatim}
## linf  =  319
## k  =  0.093
## to  =  -0.97
##  sclw = 1.95e-05 #1.95e-5 #scaler in length-weight allometry
## plw =  3.009  #power in length-weight allometry
## m50 =  4 #50% maturity
## std50 =  0.8 #std at 50% maturity
\end{verbatim}




\subsubsection{Initial values for estimated parameters}
$$\theta   =  (R_0,   \kappa,   M,  \bar{R},   \rho,  v^2,   \gamma_k,
\boldsymbol{F}_{t}, (\phi_t)_{t=1}^T, (\epsilon_t^R)_{t=1}^T)$$


	%% ro          = mfexp(theta(1));
	%% dvariable h = theta(2);
	%% m           = mfexp(theta(3));
	%% log_avgrec  = theta(4);
	%% log_recinit = theta(5);
	%% rho         = theta(6);
	%% varphi      = sqrt(1.0/theta(7));
	%% sig         = sqrt(rho) * varphi;
	%% tau         = sqrt(1-rho) * varphi;

        
        The  control  file  provides  initial  values  for  parameters
        $\theta$, the  following values  have been used  to initialise
        the model.

\paragraph{Value for $R_0$}


\paragraph{Value for $\bar{R}$}
If the  model doesn't supposed  unfished conditions at  starting year,
the recruitment at  the first year is  not $R_0$. It is  assumed to be
equal to
$$R_{t=1} = \bar{R}_{init} \exp{\epsilon_1^R}$$
In \iscam code, the recruitmnent at year $t$ is defined by
$$R_t(t)=\bar{R} \exp{\epsilon_t^R}, $$
  where $\epsilon_t^R\overset{i.i.d}{\sim} \mathcal{N}(0,\sigma_R^2)$.


\subsection{Assumption on vulnerability parameters}
When catch at age data are available, a vunlnerability curve using cubic Bsplines is fitted and currently it is assumed to be constant over time. (\com{Option 3 in selectivity option for \iscam})
It is currently not possible with \iscam to specify 0 for vulnerability, since it is expressed in log scale and because \admb behaves better with differentiable functions.





The file used for illustrating the way to use iscam on Bluefin Tuna is
\begin{verbatim}
## /home/metienne/ICCAT/ICCAT-BFT/Inputs/bfte/2012/vpa/reported/low/bfte2012.d1
\end{verbatim}

{\centering \includegraphics[width=\maxwidth]{figure/ICCAT-Selectivity2} 

}

Only 6 gears


{\centering \includegraphics[width=\maxwidth]{figure/ICCAT-SelectivityBygear} 
}




\begin{figure}
{\centering \includegraphics[width=\maxwidth]{figure/ICCAT-SelecBef80} 

}

\caption[Selectivity at age before 1980]{Selectivity at age before 1980\label{fig:SelecBef80}}
\end{figure}


\begin{figure}
{\centering \includegraphics[width=\maxwidth]{figure/ICCAT-SelecAft80} 

}
\caption[Selectivity at age after 1980]{Selectivity at age after 1980\label{fig:SelecAft80}}
\end{figure}


\subsection{First results}
\begin{verbatim}
## [1] 0.1385
\end{verbatim}
\definecolor{shadecolor}{rgb}{0.969, 0.969, 0.969}\color{fgcolor}
\begin{alltt}
\hlcomment{# print(res$A) print(res$Ahat) print(res$A_nu)}
\end{alltt}
\begin{alltt}
\hlfunctioncall{print}(res$fmsy)
\end{alltt}
\begin{alltt}
\hlfunctioncall{print}(res$msy)
\end{alltt}
\begin{verbatim}
## [1] 29631000
\end{verbatim}
\begin{alltt}
\hlfunctioncall{print}(res$bmsy)
\end{alltt}
\begin{verbatim}
## [1] 152605000
\end{verbatim}
\begin{alltt}
\hlfunctioncall{print}(res$bo)
\end{alltt}
\begin{verbatim}
## [1] 526851000
\end{verbatim}
\begin{alltt}
\hlfunctioncall{print}(res$ro)
\end{alltt}
\begin{verbatim}
## [1] 2617200
\end{verbatim}
\begin{alltt}
\hlfunctioncall{print}(res$q)
\end{alltt}
\begin{verbatim}
## [1] 7.508e-04 7.771e-07 2.755e-08 1.903e-07 1.642e-06 6.104e-06
\end{verbatim}


\begin{alltt}
\hlfunctioncall{print}(res$steepness)
\end{alltt}
\begin{verbatim}
## [1] 0.9083
\end{verbatim}

\paragraph{Selectivity}

The vulnerability matrix is estimated to 
\begin{resultz}
V=(v_{k,a})=\left(
    \begin{matrix}
      0.64 & 0.87 & 1.34 & 1.31 & 0.91 & 0.73 & 0.79 & 0.96 & 1.16 & 1.27 \\ 
      0.00 & 0.00 & 0.01 & 0.02 & 0.06 & 0.15 & 0.41 & 1.07 & 2.64 & 5.64 \\ 
      0.00 & 0.00 & 0.01 & 0.02 & 0.06 & 0.18 & 0.52 & 1.36 & 2.97 & 4.85 \\ 
      0.00 & 0.00 & 0.00 & 0.00 & 0.00 & 0.00 & 0.00 & 0.00 & 0.00 & 9.97 \\ 
      0.10 & 0.34 & 0.79 & 1.13 & 1.25 & 1.27 & 1.28 & 1.28 & 1.28 & 1.28 \\ 
      2.05 & 2.20 & 2.29 & 1.75 & 0.98 & 0.45 & 0.18 & 0.06 & 0.02 & 0.01 \\ 
      2.61 & 2.44 & 2.05 & 1.45 & 0.81 & 0.38 & 0.15 & 0.06 & 0.02 & 0.01 \\ 
    \end{matrix} 
  \right)
  \label{res:seltable}
\end{resultz}


\clearpage
\section{Simulations study}
\label{sec:simulation}
Simulations have been run to check the hability of \iscam at estimating
the parameters. The value of the parameters is chosen to be realistic in regard of the parameter estimates on BFT east data.

\subsection{Simulation model}
\iscam proposes a simulation model but in order to fully controll the simulation framework (especially the fishing mortality), a \verb+R+ code has been developped to simulate a stock dynamic according to the model specified in \iscam.
In \iscam some realistic fishing mortality is derived from the true catch and then used to simulate new catch observation. 
To avoid this back and forth procedure, a simpler version is developped: 
the fishing mortality is taken constant over year, the value being chosen to avoid stock extinction.


\subsection{Simulation framework}
The purpose of this section is to investigate the possibility for \iscam to reestimate the actual parameters. The value of the parameters have been chosen to be similar to the estimates obtained for the first run.
But the strength of the information contained in the data is obviously a key point, therefore in order to investigate the amount of noise \iscam is able to handle, different scenario have been studied.

The total variance observed on the BFTE data is estimated to $1.16$ and the proportion of the process error is set to $0.4$ which corresponds to $0.68$ for the standard deviation of the observation errors and $0.83$ for the process errors. 
This situation is worse than scenario 9 in terms of information in the data. This situation is investigated in scenario 10.

The remaining parameters used for this simulation study  are specified in \ref{tab:simPar}.		

\begin{table}[htbp]

\begin{center}
\begin{tabular}[t]{lllllll}
Scenario & Process error & Obs. error & $\tau_R$ & $\tau_I$ & Total Precision & Portion of variance $\rho$ \\ \hline
1 & low & low & 0.1 & 0.1 & 50 & 0.5\\
2 & low & med & 0.1 & 0.2 & 20 & 0.8 \\ 
3 & low & high & 0.1 & 0.5 & 3.84 & 0.96 \\ 
4 & med & low & 0.2 & 0.1 & 20 & 0.2\\
5 & med & med & 0.2 & 0.2 & 12.5 & 0.5 \\ 
6 & med & high & 0.2 & 0.5 & 3.45 & 0.86 \\ 
7 & high & low & 0.5 & 0.1 & 3.84 & 0.04\\
8 & high & med & 0.5 & 0.2 & 3.45 & 0.14 \\ 
9 & high & high & 0.5 & 0.5 & 2 & 0.5 \\ 
10 & BFTE & BFTE & 0.83 & 0.68 & 0.86 & 0.4\\ 
\end{tabular}
\end{center}
\caption{Summary of the considered scenarios in term of variance errors for the simulatio }
\end{table}


For every scenario 600 datasets have been drawn and the parameters have been reestimated. When \iscam produces a non definite positive Hessian matrix, the simulation is dropped, but this case never happens within our simulation framework.

\begin{table}[ht]
\centering
\begin{tabular}{ c  p{4cm}  p{8cm} }
  \hline
Name & Value & Notes \\ 
  \hline
\multicolumn{3}{c}{\bf Parameters be estimated}\\
$R_0$ & $exp(14.64)$ & \\
$h$ & $0.9$ & \\
$\tau_C$&0.1 & This parametere is fixed in \iscam not estimated\\
$\tau_A$ & 0.1 & \\
$v_{k,a}$&  see \ref{res:seltable}  & the  selectivity table  is taken
according to the results obtained on BFT \\
$q_k$ & $10^{-5}$ & Same value for all gear\\
\multicolumn{3}{c}{\bf Other quantities  of interest (fixed parameters
  or derived from parameters)}\\
$f$&  0.2  & the  instant fishing mortality, chosen to be constant over year\\
$\Phi_E$ & $317$ & \\
$M_a$ & $\left(0.490,0.24, 0.24, 0.24, 0.240,\right.$ $\left. 0.20, 
0.175 , 0.15, 0.125, 0.10\right)$&\\
$m_{50}$& $4$&\\
$\sigma_{50}$& $0.8$&\\
$s_0$ &  $0.113$ & fixed  according to $R_0$ and  $h$ using
$s_0= \kappa/\Phi_E$ and $\kappa=4h/(1-h)$ \\
$\beta$ & $4.84\, 10^{-08}$ & fixed using $\beta=(\kappa -1)/(R_0*\phi_E)$\\
\hline
\end{tabular}
\caption{Leading parameters}
\label{tab:simPar}
\end{table}



\subsection{Results on leading parameters}
\subsubsection{Scenario 10 - BFTE like parameters value}


Table \ref{tab:quant} sums up the resstimated values through the obtained quantiles. The extremal results are very far from the simulated values 
but the extremal behaviour is very uncommon.   Removing only the 2  most extremals percents of the estimations produce  robust estimates .


\begin{table}[ht]
\begin{center}
\begin{tabular}{rrrrrrrrr}
  \hline
 & $log(R_0)$ & $\log(R_{init})$ & $\varphi$ & $\tau_I$ & $\tau_R$ & $MSY$ & $Fmsy$ & $Bmsy$ \\ 
  \hline
0\% & 14.43 & 9.09 & 0.94 & 0.59 & 0.73 & 16.89 & 0.08 & 18.45 \\ 
  1\% & 14.54 & 12.13 & 1.01 & 0.64 & 0.78 & 17.00 & 0.13 & 18.56 \\ 
  2.5\% & 14.59 & 12.50 & 1.03 & 0.65 & 0.80 & 17.05 & 0.13 & 18.60 \\ 
  5\% & 14.63 & 12.67 & 1.05 & 0.67 & 0.82 & 17.09 & 0.14 & 18.64 \\ 
  50\% & 14.84 & 14.04 & 1.20 & 0.76 & 0.93 & 17.30 & 0.15 & 18.86 \\ 
  95\% & 15.09 & 15.98 & 1.39 & 0.88 & 1.07 & 17.54 & 0.15 & 19.13 \\ 
  97.5\% & 15.19 & 17.73 & 1.45 & 0.92 & 1.13 & 17.64 & 0.15 & 19.21 \\ 
  99\% & 15.28 & 18.68 & 1.51 & 0.96 & 1.17 & 17.70 & 0.15 & 19.32 \\ 
  100\% & 29.99 & 19.99 & 1.67 & 1.05 & 1.29 & 32.08 & 0.16 & 34.31 \\ 
   \hline
\end{tabular}
\end{center}
\caption{Empirical quantiles obtained by reestimation on simulated data for the main parameters}
\label{tab:quant}
\end{table}


The reestimated values are realistically closed from the simulated values, but \iscam seems to produce systematically biased estimates. The figures \ref{fig:sim} illustrates the bias in the reestimation.
 \begin{figure}
  \begin{subfigure}[b]{\textwidth}
	\begin{subfigure}[b]{0.45\textwidth}
	\includegraphics[width=0.9\textwidth]{figure/ICCAT-simrinit}
	\caption{ Histogram of the reestimation of the initial recruitment $log(R_{init})$ }
	\end{subfigure}\hfill
	\begin{subfigure}[b]{0.45\textwidth}
	\includegraphics[width=0.9\textwidth]{figure/ICCAT-simro}
	\caption{ Histogram of the reestimation of unifshed recruitment $log(R_0)$}
	\end{subfigure}
  \end{subfigure}
  \begin{subfigure}[b]{\textwidth}
\begin{subfigure}[b]{0.45\textwidth}
  \includegraphics[width=0.9\textwidth]{figure/ICCAT-simh}
  \caption{ Histogram of the reestimation of the steepness $h$ }
 \end{subfigure}\hfill
 \begin{subfigure}[b]{0.45\textwidth}
  \includegraphics[width=0.9\textwidth]{figure/simvarphi}
  \caption{ Histogram of the reestimation of the total variance $\varphi$}
 \end{subfigure}
  \end{subfigure}
 \caption{Summary of the reestimated values. The red lines is the true value used for the simulation. The 5 most extremal percent have been droped.}
 \end{figure} 



\subsection{
\subsection{Discussion on simulation results}
The estimation procedure proposed in \iscam seems to systematically under estimate the initial recruitment  and to over estimate the recruitment in unfished conditions, the total variance and the steepness.
This may arrise from the fact that the simulated model used an actual stock recruitment relationship but \iscam used a two step procedure. It first estimates the recruitment as random effects centred on a parametre $\bar{R}$ with a a variance $\tau_R$. In a second phase, a SR relationship is adjusted using the estimated recruitment and the same variance is used to control the deviation from the stock recruitment model.
Therefore $\tau_R$ is used to represent two different kind of deviations and may produce biased estimation.


\section{Complete results on Bluefin tuna data - East stock}
\label{sec:results}

\subsection{Maximum likelihood approach}
Report the MLE for R0, Rinit h BMSY,FMSY B[2011]/BMSY, F[2011]/FMSY for each scenario
\begin{table}
 
\end{table}




{\centering \includegraphics[width=\maxwidth]{figure/ICCAT-KobePlot} 
}

\subsection{Retrospective analysis}
A retrospective analysis has been run on the data set corresponding to inflated-high  catch scenario, droping up to 8 years.
Figure 


\begin{figure}
\begin{subfigure}[b]{\textwidth}
\includegraphics[width=\maxwidth]{figure/ICCAT-SBTRetro}  
\caption{Estimation of the stock biomass with retrospective analysis}
\end{subfigure}
\begin{subfigure}[b]{\textwidth}
\includegraphics[width=\maxwidth]{figure/ICCAT-BmsyEvolRetro}  
\caption{Evolution of the estimated value for $B_{MSY}$ according to the last year in the retrospective analysis}
\end{subfigure}
\caption{Retrospective analysis, droping 1 to 8 years}
\end{figure}



\subsection{Monte Carlo approach}

\paragraph{Convergence}

\begin{figure}[htbp]
 \begin{subfigure}[b]{0.45\textwidth}
 \includegraphics[width=\textwidth]{figure/ICCAT-Conv1}
  \caption{Stock status at the last year}
                \label{fig:mcmcstock}
  \end{subfigure}
 \begin{subfigure}[b]{0.45\textwidth} 
 \includegraphics[width=\textwidth]{figure/ICCAT-Conv2}
\caption{Fishing status at the last year}
                \label{fig:mcmcfish}
  \end{subfigure}
  \caption{Graph for MCMC chains}
  \label{fig:mcmcdiag}
 \end{figure}

\paragraph{Stock Statut}
{\centering \includegraphics[width=\maxwidth]{figure/ICCAT-PostStatus} 
}



\section{Discussion}
\label{sec:concl}

The results obtained with \iscam are consistent with the results described in \cite{tuna2012} and obtained using a virtual population analysis.

Furthermore, as \iscam used the stock recruitment relationship in the last phase of estimation, the recruitment may be almost freely estimated as in a VPA approach. 


Some of the parameters have to be fixed and can't be estimated with the data available for this study.
\iscam may be more finely tuned using more expertise on the studied stock; this expertise may be included as a prior for example.
Furthermore, the mortality is a key parameter and has been chosen constant throughout ages, but recent development of \iscam should allow to use a mortality rate varying with age and produce results closer from the previous stock assesment.
\iscam also proposes some prediction of the evolution of the stock according to fixed TAC policy. This code to use this possibility of \iscam has been developped but has not been presnted here.

%The following appendix aim at giving some basic information to use the code produced during the analysis presented in this report.
%\verb+admb+ and \verb+iSCAM+ are assumed to be installed, a \verb+PATH+ variable \verb+ADMB_HOME+ exists and contains the path to \verb+admb+ directory. In addition a \verb+PATH+ variable, called \verb+ISCAM+ contains the path to \verb+iscam+ executable.   
\section{R codes produced and how to use them}
All the codes and results described  or used in this report are freely
available on \url{https://github.com/MarieEtienne/ICCAT-BFT}. 

\iscam  has  been  developped  by  Steven  Martell,  is  available  on
\url{https://github.com/smartell/iSCAM} and described in \cite{Martell11}.

Using \iscam  required to write  some several datafiles as  inputs. To
produce an automatic process and in a context of reproducible science,
some \rcode codes  are used to translate a vpa  format file \verb+*.d1*+ in
inputs     file      for     \iscam.     The     main      file     is
\verb+sources/setISCAMFiles.R+  and required  the directory  where the
*.d1 file  is located. It  generates the input  files for \iscam  in a
directory based on the input  directory. With the current organisation
of the on GitHub, the input files are in \verb+ICCAT-BFT/Input/bfte/2012/...+, so
a directory  \verb+bfte/2012/..+ will  be created  in \verb+ICCAT-BFT+
and \iscam can belaunched from this directory.

\section{How to run the codes to reproduce the analysis}

%\appendix
\bibliographystyle{apalike}
\bibliography{biblio}

\end{document} 
