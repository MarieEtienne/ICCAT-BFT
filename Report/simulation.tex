Simulations have been run to check the hability of \iscam at estimating
the parameters. The value of the parameters is chosen to be realistic in regard of the parameter estimates on BFT east data.

\subsection{Simulation model}
\iscam proposes a simulation model but in order to fully controll the simulation framework (especially the fishing mortality), a \verb+R+ code has been developped to simulate a stock dynamic according to the model specified in \iscam.
In \iscam some realistic fishing mortality is derived from the true catch and then used to simulate new catch observation. To avoid this back and force procedure, a simpler version is developped. The fishing mortality is taken constant over year, the value being chosen to avoid stock extinction.
\subsection{Simulation framework}

Thousand datasets have been drawn and the parameters have been reestimated. The simulated value are specified in \ref{tab:simPar}.		

\begin{table}[ht]
\centering
\begin{tabular}{ c  p{4cm}  p{8cm} }
  \hline
Name & Value & Notes \\ 
  \hline
\multicolumn{3}{c}{\bf Parameters be estimated}\\
$R_0$ & $exp(14.64)$ & \\
$h$ & $0.9$ & \\
$\rho$ & 0.4 & \\
$\varphi$ & 1 & \\
$\tau_R$ & $0.774=\sqrt{1-\rho} * \varphi$ & \\
$\tau_I$ & $0.632=\sqrt{\rho} * \varphi$ & \\
$\tau_C$&0.1 & This parametere is fixed in \iscam not estimated\\
$\tau_A$ & 0.1 & \\
$v_{k,a}$&  see \ref{res:seltable}  & the  selectivity table  is taken
according to the results obtained on BFT \\
$q_k$ & $10^{-5}$ & Same value for all gear\\
\multicolumn{3}{c}{\bf Other quantities  of interest (fixed parameters
  or derived from parameters)}\\
$f$&  0.2  & the  instant fishing mortality, chosen to be constant over year\\
$\Phi_E$ & $317$ & \\
$M_a$ & $\left(0.490,0.24, 0.24, 0.24, 0.240,\right.$ $\left. 0.20, 
0.175 , 0.15, 0.125, 0.10\right)$&\\
$m_{50}$& $4$&\\
$\sigma_{50}$& $0.8$&\\
$s_0$ &  $0.113$ & fixed  according to $R_0$ and  $h$ using
$s_0= \kappa/\Phi_E$ and $\kappa=4h/(1-h)$ \\
$\beta$ & $4.84\, 10^{-08}$ & fixed using $\beta=(\kappa -1)/(R_0*\phi_E)$\\
\hline
\end{tabular}
\caption{Leading parameters}
\label{tab:simPar}
\end{table}

\subsection{Results on leading parameters}
When \iscam produces a non definite positive Hessian matrix, the simulation is dropped. It occurs NOMBRENOMBRE on the thousands simulations (corresponding to  NOMBRENOMBRE $\%$).


\begin{table}[ht]
\begin{center}
\begin{tabular}{rrrrrrrrr}
  \hline
 & ro & rinit & varphi & tau\_I & tau\_R & MSY & Fmsy & Bmsy \\ 
  \hline
0\% & 14.43 & 9.09 & 0.94 & 0.59 & 0.73 & 16.89 & 0.08 & 18.45 \\ 
  1\% & 14.54 & 12.13 & 1.01 & 0.64 & 0.78 & 17.00 & 0.13 & 18.56 \\ 
  2.5\% & 14.59 & 12.50 & 1.03 & 0.65 & 0.80 & 17.05 & 0.13 & 18.60 \\ 
  5\% & 14.63 & 12.67 & 1.05 & 0.67 & 0.82 & 17.09 & 0.14 & 18.64 \\ 
  50\% & 14.84 & 14.04 & 1.20 & 0.76 & 0.93 & 17.30 & 0.15 & 18.86 \\ 
  97.5\% & 15.19 & 17.73 & 1.45 & 0.92 & 1.13 & 17.64 & 0.15 & 19.21 \\ 
  99\% & 15.28 & 18.68 & 1.51 & 0.96 & 1.17 & 17.70 & 0.15 & 19.32 \\ 
  100\% & 29.99 & 19.99 & 1.67 & 1.05 & 1.29 & 32.08 & 0.16 & 34.31 \\ 
   \hline
\end{tabular}
\end{center}
\caption{Empirical quantiles obtained by reestimation on simulated data for the main parameters}
\end{table}




%\begin{figure}
%  \includegraphics[width=8cm]{R0estimation}
%  \includegraphics[width=8cm]{tau_Restimation}
%\end{figure} 


%%\begin{figure}
%%  \includegraphics[width=8cm]{SteepnessEstimation}
%%\end{figure}

%%\begin{figure}
%%  \includegraphics[width=8cm]{BmsyEstimation}
%%\end{figure}


%%\begin{figure}
%%  \includegraphics[width=8cm]{fmsyEstimation}
%%\end{figure}


%%\begin{figure}
%%  \includegraphics[width=8cm]{StockStatusEstimation}
%%\end{figure}


%%\begin{figure}
%%  \includegraphics[width=8cm]{KobeplotEstimation}
%%\end{figure}



\subsection{Discussion on simulation results}
