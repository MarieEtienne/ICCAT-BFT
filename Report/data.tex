The presented  work ocus on  Bluefin Tuna  east stock data.  There are
seven abundance indices and the total commercial catch available. 





\subsection{Abundance indices}

Seven abundance indices are available.   the oldest one starts in year
1950,  the last  indices considered  stop in  2011. Three  indices are
assumed to be proportional to  the total number of tuna (\verb+SM_TP+,
\verb+LL_JP1+,  \verb+LL_JP2+),  the four  others  are  assumed to  be
proportionnal to the total biomass (expressed as weight).

\begin{figure}
{\centering \includegraphics[width=0.7\maxwidth]{figure/ICCAT-Abundance} 
}
\caption{Scaled abundance indices }
\end{figure}




\subsection{Available catch}
The  total catch  in  numbers  are given  from  1950  to 2011.  Figure
\ref{fig:TotCatch} illsutrates the evolution of the ctach across years.
\begin{figure}
\centering \includegraphics[width=0.7\maxwidth]{figure/ICCAT-Catch} 
\caption{Catch  splitted by  age over  the whole  period. Area  of red
  circles is proportionnal to the Catch}
\label{fig:Totcatch}
\end{figure}


\subsection{Data on Selectivity}
Catch at age  data are available for the commercial  fisheries and for
six of the seven abundance  indices. It is specified that \verb+NW_PS+
is proportionnal to last age class (more than 10 years old tuna).




To investigate the  change in selectivity, we can  look at composition
of the catch per gear and per year
\begin{figure}
{\centering \includegraphics[width=\maxwidth]{figure/ICCAT-Selectivity11} 
}
\caption{Catch splitted by age and gear over the whole period. Area of the
  circles is proportionnal to the Catch}
\end{figure}




