This section  sums up the principle  aspects of \iscam used  on BFTE.
\subsection{Latent process Level}
\subsubsection{Population model}
\iscam is an age structured model.  Fish population is split into age classes from
age $sage$ to age $nage$. Let us denote by $\Nt = (N_{1,t}, \ldots, N_{A,t})$ the number
of individuals in every class age $a$ ($a\in [1,A]$) in year $t$ ($t\in [1;T]$).

\begin{gather}
  % \mbox{Initial state of the population}\\
  \mbox{Population dynamic after year syr (t>1)}\\
  N_{a,t}= \left\lbrace 
    \begin{array}{l}
      R_t, \quad a=1\\
      N_{a-1,t-1} \exp(-Z_{t-1, a-1}), \quad a\in [2;A-1]\\
      N_{a-1,t-1} \exp(-Z_{t-1, a-1}) + N_{a,t-1} \exp(-Z_{t-1, a}), \quad a=A,\\
    \end{array}  \right. \\
\end{gather}
 where $Z_{a,t}$  stands for the total  mortality rate at age  $a$ in
  year $y$ and $R_t$ is the recruitment at year $t$.




\begin{gather}
\mbox{Recruitment after year syr (t>1)}\\
R_t= \frac{s_0 B_{t-1}}{1+\beta B_{t-1}} \exp{\epsilon^R_t}, \\
\epsilon^R_t\underset{i.i.d}{\sim}\mathcal{N}\left(-\frac{\tau_R^2}{2}, \tau_R^2\right),\\
\label{eq:iscamBev}
\end{gather} 

\iscam assumes a stock recruitment relationship which links the mature biomass at
time $t$ $B_t$ and the recruitment at time $t+1$. The two options for this stock recruitment model are  Beverton and Holt
(BH) or Ricker (R) model. The presented work used the BH model.  \\

To circumvent estimation issues, \iscam uses a trick and  during the first estimation phases, the recruitment
model is specified by the following equation:
\begin{gather}
\label{eq:iscamRec}  R_t =\bar{R}e^{\omega_t},
\end{gather}
where  $\omega_t\overset{i.i.d}{\sim}  \mathcal{N}(0,  \tau^2_R)$  and
$\bar{R}$ is the average recruitment. 

This
approach  in the  early  phases is  close to  the  VPA from  the
recruitment  point  of  view,  since   it  doesn't  assume  any  stock
recruitment relationship. 


During the last phase, the  BH relationship is estimated. Turning off this estimation is possible and would allow to base management evaluation strategies on an average annual recruitment.


Special attention is required for the first year of the study. The recruitment parameter $R_{init}$ during this first year has to be estimated.
If the stock is considered in unfished condition, then $R_{init}$ is assumed to be equal to $R_0$, otherwise $R_{init}$ is specifically estimated. 

\begin{gather}
\mbox{Definition of mature biomass}\\
B_t= \sum_{a=1}^A N_{t,a} f_a,
\end{gather} 
$f_a$ being the fertility at age $a$. 

$f_a$ doesn't depend  on year $t$, fertility at  age is considered
  as fixed over years

  
  
\begin{gather}
  \mbox{Definition of the total mortality rate ate age}\\
  Z_{t,a}= M_a + \sum_{k=1}^K F_{k,t} v_{k, t,a}, \quad \mbox{gear }k,
  \mbox{class }a, \mbox{ at year }t.   \\
\end{gather} 
In  the  BFT  data  available  there is  only  one  time  series  for
 catch (catch are  not split among different  gears), considering that
 the gear index corresponding to the 
commercial fisheries equals one, the previous expression may be simplified as~:

\begin{gather}
  \mbox{Definition of total mortality rate ate age}\\
  Z_{t,a}= M_a + F_t v_{1, t,a}, \quad \mbox{gear } 1,
  \mbox{class }a, \mbox{year }t,   \\
\end{gather} 

where $F_t$ is the instantaneous  fishing mortality and $v_{1,t,a}$ is
the vulnerability for gear $1$, in year $t$ for age class a.

The weight at age is derived from an allometric relationship between length and weight, and from the von Bertalanffy growth equation to link age class and length.
The fertility at age is assumed to follow a logistic function with parameters $\mu_{mat 50\%}$ and $\sigma_{mat 50\%}$ 
\begin{gather}
  \mbox{Life trait specification}\\
l_a = l_\infty \left( 1- \exp{\left(-k\left( a - t_0\right)\right)}\right) \\
w_a = a_w\, l_a^{b_w} \\
f_a = w_a \frac{1}{1+\exp{ \left(\frac{\mu_f - a}{\sigma_f}\right)}}\\
  \end{gather}
  


\subsubsection{Vulnerability model}
\label{sec:vulnerability}
\iscam    allows     to    specify    different    form     for    the
selectivity/vulnerability. The vulnerability  may either be completely
specified by the user, or when  age composition data are available the
vulnerability may be  inferred. In the latter case,  different form of
selectivity curves may be specified:  for example the selectivity may be
chosen as a logistic function of age, or even, with a more flexible approach, using B-spline.  The B-splines
functions may  even model a  variation across years. In  this work,
only logistic or simple Bspline has been used to circumvent estimation issues and avoid over-parametrization.


\subsection{Observation Level}
\subsubsection{Age composition data}

Following  \cite{Schnute+95}, \iscam  uses by  default a  multivariate
logistic  function  for  age  composition data.  It  is  assumed  that
$p_{atk}$ which is the  proportion of fish of age $a$  in year $t$ for
gear $k$, is drawn from the following distribution (gear $k$ is omitted
for clarity) which is defined thanks to a latent variable $X_{at}$
\begin{gather}
\epsilon^{A}_{at} \overset{i.i.d}{\sim} \mathcal{N}(0,\tau^2_A)\nonumber \\ 
X_{at}       =       \log{\mu_{at}}       +       \epsilon_{at}^A       -
\frac{1}{A}\sum_{a=1}^A\left(\log{\mu_{at}} + \epsilon_{at}^A \right) \nonumber \\
e^{X_{at}} = \frac{\mu_{at}e^{\epsilon_{at}^A}}{ \left(\prod_{a=1}^A \mu_{at}e^{\epsilon_{at}^A}\right) ^{1/A}}\nonumber\\
p_{at} = \frac{e^{X_{at}} } {\sum_{a=1}^A e^{X_{at}} }\\
\label{equa:multivar}
\end{gather}
The multivariate logistic distribution  avoids the drawback of a very high
precision when using a classical multinomial distribution.

\subsubsection{Abundance indices}
The total vulnerable biomass at year $t$ for gear $k$ is defined by
\begin{gather}
V_{k,t}=\sum_{a=1}^A N_{t,a} e^{-\lambda_{k,t} Z_{t,a}} v_{k,a} w_a,
\end{gather}
 where $\lambda_{k,t}$ is the fraction  of the mortality to adjust for
 survey timing, it is specified by the user.

 
\begin{gather}
I_{k,t} = q_k V_{k,t} \exp{\epsilon_{k,t}^I}\\
\epsilon_{k,t}^I \overset{i.i.d}{\sim} \mathcal{N}(0,\tau_I^2)\nonumber
\end{gather}
\subsubsection{Catch}
Catch  with  gear  $k$  in  year  $t$  is  denoted  by  $C_{k,t}$  and
defined using Baranov Catch equations by
\begin{gather*}
\hat{C}_{k,t}   =   \sum_{a=1}^A   \frac{N_{t,a}   w_a   F_{k,t}   v_{k,t,a}
  (1-e^{-Z_{t,a}}) }{Z_{t,a}},\\
\end{gather*}

Since all catch are aggregated and treated as one gear, this equation simplifies in 
\begin{gather}
\hat{C}_{t}   =   \sum_{a=1}^A   \frac{N_{t,a}   w_a   F_{t}   v_{1,t,a}
  (1-e^{-Z_{t,a}}) }{Z_{t,a}},\\
C_{t} = \hat{C}_{t} \exp{\epsilon_{t}^C}, \quad \epsilon_{t}\overset{i.i.d}{\sim}\mathcal{N}(0, \tau^2_C)\nonumber
\end{gather}

\subsection{Summary of the symbols in \iscam}

Table  \ref{tab:parameters}  and  \ref{tab:symbols}  recall  the  main
symbols used in this report and give their corresponding name in \iscam. Further 
description is available  in \cite{Martell11}. Some of  the names have
been changed to unify the notation  standards. When the name is shared by this
 this paper and \iscam users Guide, nothing is specified.


\begin{table}[ht]
\centering
\begin{tabular}{p{3cm}p{4cm}p{8cm}}
  \hline
Parameters & Name in \iscam & Signification \\ 
  \hline
$\tau_C$& $\sigma_C$ & Standard deviation in observed catch - This parameter is not estimated but set to a given value\\
$\tau_A$ & $\tau$,  mentioned only through its estimation  on page 21
of \cite{Martell12} & standard deviation in the multivariate logistic function \\
$v_{k,a}$& &vulnerability for gear $k$ at age $a$\\
$R_0$ && Recruitment in unfished conditions\\
$h$ && Steepness\\
$q_k$ & & Catchability coefficient for gear $k$\\
$\rho$ & & the proportion of total variation allocated to the variance
of the survey index\\
$\varphi^2$ & $1/\mathcal{V}^2$& total variance $\varphi^2=\tau^2_R+\tau^2_I$\\
$\tau_I=\varphi*\sqrt{\rho}$ & $\sigma$ & Standard deviation in the survey index\\
$\tau_R=\varphi*\sqrt{1-\rho}$ & $\tau$& Standard deviation in process
errors\\
   \hline
\end{tabular}
\caption{Leading parameters}
\label{tab:parameters}
\end{table}

\subsubsection{Other symbols}

\begin{table}[ht]
\centering
\begin{tabular}{p{3.5cm}p{3.5cm}p{8cm}}
  \hline
Name & Name in \iscam & Signification \\ 
  \hline
$A$ & & last class age\\
$\kappa=4h / (1-h)$ & & Goodyear compensation ratio for BH\\
$s_0= \kappa/\Phi_E$ & & BH relationship\\
$\beta=(\kappa-1) / (R_0 \Phi_E)$ &&BH relationship\\
$\epsilon_t^R$ & $\delta_t$& recruitment errors\\
$\epsilon_{t,a}^A$ & $\eta_{t,a}$& errors in the multivariate logistic \\
$\epsilon_{k,t}^I$ & $\epsilon_{k,t}$& residuals in abundance survey\\
$\epsilon_t^C$ & $\eta_{k,t}$& residuals in catch data\\
$\bar{R}$ &  &Average recruitment  used in  first estimation  phases as
mentioned in equation \ref{eq:iscamRec}\\
$B_t$ & &Spawning biomass \\
$N_{a,t}$&& Number of individuals of age a in year t\\
$f_a$ && fecundity at age $a$\\
$M_a$ && instantaneous mortality rate at age $a$\\
$F_{k,t}, F_{t}$&& Instantaneous fishing mortality  for gear k in year
t, since  there is only one  commercial gear considered, index  $k$ is
mostly omitted\\
$Z_{a,t}$& &Instantaneous total mortality for age a in year t\\ 
$X_{a,t}$&   not  named   in  \iscam&   Quantities  involved   in  the
multivariate logistic\\
$\mu_{a,t}$ &  & unnormalized proportion at age\\
& $\widehat{p_{a,t}}=\mu_{a,t}/\sum_a \mu_{a,t}$& proportion at age\\
$p_{a,t}$& & Observed proportion at age\\
$V_{k,t}$& & Total vulnerable biomass for gear k in year t\\
${w}_{a}$& &Weight at age a\\
   \hline
\end{tabular}
\caption{List of symbols used in \iscam}
\label{tab:symbols}
\end{table}
